\chapter{Kurzfassung}

Der Kongruenzhüllenalgorithmus ist ein zentraler Algorithmus für Entscheidungsprozeduren für Theorembeweiser. In diesem Zusammenhang ist es wichtig, vertrauenswürdige Implementationen zu haben, deren Korrektheit formal bewiesen ist. Diese Arbeit befasst sich mit der formalen Implementierung und Verifizierung eines Kongruenzhüllenalgorithmus im interaktiven Theorembeweiser Isabelle/HOL. Die Implementation basiert auf dem Union-Find-Algorithmus. In dem Kontext von Theorembeweisern ist es notwendig, ein Zertifikat zu berechnen, das beweist, dass zwei Terme kongruent sind. Dafür gibt es die \emph{explain} Operation. Die Algorithmen dieser Arbeit basieren auf dem Artikel von Nieuwenhuis and Oliveras \cite{Nieuwenhuis}.
Als Erstes wird die \emph{explain} Operation für Union-Find implementiert mit Beweisen für Korrektheit und Terminierung. Dies erweitert die Formalisierung der Union-Find-Datenstruktur in Isabelle/HOL von Lammich \cite{unionfind-isabelle}. Dann wird der Kongruenzhüllenalgorithmus implementiert und verifiziert. Zuletzt wird die \emph{cc\_explain} Operation für den Kongruenzhüllenalgorithmus mit einem Terminationsbeweis beschrieben. Der Korrektheitsbeweis für \emph{cc\_explain} ist noch offen für künftige Arbeiten.