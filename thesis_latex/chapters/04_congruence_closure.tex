% !TeX root = ../main.tex
% Add the above to each chapter to make compiling the PDF easier in some editors.

\chapter{Congruence Closure with Explain Operation}\label{chapter:congruence_closure}

\section{Input equations}

We now consider not only equations between constants, but also equations containing function symbols. Each function symbol is associated with an arity, which is the number of parameters it accepts. Function symbols with arity 0 are constants.

Arbitrary equations can be transformed to equations of depth at most 2, and with only one function of arity 2. This is done by currying and by introducing new constant symbols. See \cite{Nieuwenhuis2} for a detailed explanation of how this is done. In order to understand this thesis, it is irrelevant to know how the transformation is done, it is just important to know that the result is a set of equations between constant symbols and one specific function symbol $F$ of arity 2. The congruence closure of these transformed equations is equal to the congruence closure of the original equations, therefore from this point on, we will only consider the transformed equations. Therefore there are two types of input equations for the congruence closure algorithm: either $a = b$ or $F(a,b) = c$ where $a$, $b$ and $c$ are constant symbols.

The data structure we use for these equations is the one described in Subsection \ref{subsection:datastructure}.

\section{Implementation}

For the implementation of the congruence closure algorithm, we follow the description in the paper. \cite{Nieuwenhuis} As before, the optimizations of path compression and considering the sizes of the representative classes are left out. These optimizations are not relevant for the correctness of the algorithm, and they could later be added to a refinement of the algorithm.

The algorithm uses a modified version of the union find algorithm for maintaining the equivalence classes between the constant elements, and it has a few additional data structures for storing equations containing the function symbol $F$. The following sections describe the modified union-find algorithm and then the congruence closure algorithm, as well as their correctness proofs.

\subsection{Modified Union Find Algorithm}

In order to implement an explain operation with a reasonable runtime for the congruence closure data structure, the paper \cite{Nieuwenhuis} introduces an alternative union-find algorithm. Additionally to the union-find forest, there is a new data structure, the \emph{proof forest}, i.e. a forest which has as nodes the variables, and as edges the unions that were made. Each time \lstinline{ufa_union} is called on the union-find forest, the proof forest is modified with \lstinline{add_edge}. In order to avoid the creation of cycles, redundant unions are ignored.

\subsubsection{add\_edge}
\label{subsubsection:addedge}

The proof forest has directed edges, and for each equivalence class there is a representative node, where all the edges are directed towards. To keep this invariant, each time and edge from $e$ to $e'$ is added, all the edges on the path from the root of $e$ to $e$ are reversed.
In the implementation, the proof forest is represented by a list which stores the parent of each node, exactly as in the union-find list. My implementation for adding an edge, which corresponds to the \lstinline{union} operation, is the following:

\begin{lstlisting}
function (domintros) add_edge :: "nat list ⇒ nat ⇒ nat ⇒ nat list"
  where
"add_edge pf e e' = (if pf ! e = e
                        then (pf[e := e'])
                        else add_edge (pf[e := e']) (pf ! e) e)"
  by pat_completeness auto
\end{lstlisting}

We can show that \lstinline{add_edge e e'} terminates, if the invariant \lstinline{ufa_invar} holds for the proof forest and $e$ and $e'$ do not belong to the same equivalence class.

\begin{lstlisting}
lemma add_edge_domain:
  assumes "ufa_invar l" "rep_of l e ≠ rep_of l e'"
  shows "add_edge_dom (l, e, e')"
\end{lstlisting}

\begin{proof}
It can be proven by induction on the length of the path $p$ from the root of $e$ to $e$.

In the base case there is only one node in the path, therefore $e$ must be equal to its root, therefore $pf ! e = e$, and the algorithm terminates immediately.

In the other case $e$ is not a root, then there is a path $p'$ from the root to the parent of $e$ which is shorter than the path from the root to $e$. The path $p'$ is also present in the $pf[e := e']$, because the path does not contain $e$. Also, the representative of $e$ in $pf[e := e']$ is equal to the representative of $e'$, and the representative of the parent of $e$ is still the old representative of $e$, therefore they are not in the same representative class, and we can apply the induction hypothesis and conclude that the recursive call terminates, therefore the function terminates.
\end{proof}

In order to prove the correctness of \lstinline|add_edge|, we show that \lstinline{ufa_union l x y} and \lstinline{add_edge l x y} have the same behaviour, from which we can conclude that the equivalence classes of the union-find forest are the same as those of the proof forest. The theory \lstinline{Union_Find} from the AFP\cite{Sep} already provides the following lemma for \lstinline{ufa_union}:

\begin{lstlisting}
lemma ufa_union_aux:
  "rep_of (ufa_union l x y) i =
    (if rep_of l i = rep_of l x then rep_of l y else rep_of l i)"
\end{lstlisting}

We can show a similar lemma for \lstinline{add_edge}:

\begin{lstlisting}
lemma rep_of_add_edge_aux:
   assumes "rep_of l x ≠ rep_of l y"
   shows "rep_of (add_edge l x y) i =
     (if rep_of l i = rep_of l x then rep_of l y else rep_of l i)"
\end{lstlisting}

The additional assumption \lstinline{rep_of l x ≠ rep_of l y} does not cause problems, because \lstinline{add_edge} is only executed by the congruence closure algorithm when \lstinline{rep_of l x ≠ rep_of l y}.

\begin{proof}
We already showed that the function terminates, therefore we can prove it by induction on \lstinline{add_edge}.

The proof uses various lemmas about the behaviour of \lstinline{rep_of} after a function update, which depends on where the function update was, and which element we want to find the representative of. These lemmas are proven by analysing how the paths in the forest change after a function update, which corresponds to adding a path in the forest. With these lemmas the induction is easily proven.
\end{proof}

We also show that \lstinline|add_edge| has the expected behaviour, which is that it reverses all the edges from the root of $e$ to $e$, and it adds an edge from $e$ to $e'$, i.e. after \lstinline|add_edge| the forest contains a path from $e'$ to the representative of $e$, which is the path which was there before \lstinline|add_edge|, but reversed and with one added edge between $e$ and $e'$. \lstinline|rev| is a function which reverses a list, and \lstinline|path_to_root| is the function described in Subsection \ref{subsubsection:path-to-root}. The proof can be shown by induction on \lstinline|add_edge|.

\begin{lstlisting}
lemma add_edge_correctness:
  assumes "ufa_invar pf"
    "rep_of pf e ≠ rep_of pf e'"
  shows "path (add_edge pf e e') e' ([e'] @ rev (path_to_root pf e)) (rep_of pf e)"
\end{lstlisting}

The proof forest has a similar structure as the union-find forest, therefore we prove that \lstinline|add_edge| preserves the \lstinline|ufa_invar| invariant from Section \ref{section:uf-isabelle}. This allows us to apply all the lemmas that were proven for the union-find forest also to the proof forest.

\begin{lstlisting}
lemma add_edge_ufa_invar_invar:
  assumes "ufa_invar l"
    "rep_of l e ≠ rep_of l e'"
  shows "ufa_invar (add_edge l e e')"
\end{lstlisting}

\begin{proof}
In order to prove this lemma, we show another lemma which states that the invariant holds after a function update if the update does not cause the formation of a cycle. Then we show that each function update of \lstinline{add_edge} does not form a cycle.
\end{proof}

\subsubsection{add\_label}

Additionally, each edge is labeled with the input equation or the input equations which caused the adding of this edge. This is not necessary for the union-find algorithm by itself, but it will be needed by the \lstinline{explain} operation for congruence closure. There are two possible types of labels: either an equation $a = b$ was input, or two equations of the type $F(a_1, a_2) = a$ and $F(b_1, b_2) = b$, where $a_1$ and $b_1$ were already in the same equivalence class before this union, as well as $a_2$ and $b_2$. In both theses cases a union between the equivalence classes of $a$ and $b$ must be made. The labeling is implemented by using an additional list, which at each index contains the label of the outgoing edge, or \lstinline{None} if there is no outgoing edge. It is similar to the associated unions list of union-find, but it contains directly the labels instead of an index to another list.

The labels have the type \lstinline{pending_equation}, which can be either one or two equations.

\begin{lstlisting}
datatype pending_equation = One equation
  | Two equation equation
\end{lstlisting}

The name \lstinline{pending_equation} derives from the fact that it is also the type of the elements of the pending list, which will be described in the next section. Theoretically this allows also for invalid equations for example two equations of the type $a = b$ and $c = d$, but we will prove in the next sections that the equations in the labels list are always either \lstinline{One} ($a = b$) or \lstinline{Two} ($F(a_1, a_2) = a$) ($F(b_1, b_2) = b$).

Each time an edge gets added to the proof forest, the labels need to be updated as well. The function \lstinline{add_label} adds a label to the new edge, and modifies the labels for the edges which are modified by \lstinline{add_edge}:

\begin{lstlisting}
function (domintros) add_label :: "pending_equation option list ⇒ nat list ⇒ nat ⇒ pending_equation ⇒ pending_equation option list"
  where
"add_label pfl pf e lbl =
    (if pf ! e = e
        then (pfl[e := Some lbl])
        else add_label (pfl[e := Some lbl]) pf (pf ! e) (the (pfl ! e)))"
  by pat_completeness auto
\end{lstlisting}

Similarly to the \lstinline{path_to_root} function, \lstinline{add_label} has the same recursive calls as \lstinline{rep_of}, therefore it has the same domain.

\begin{lstlisting}
lemma rep_of_dom_iff_add_label_dom:
  "rep_of_dom (pf, y)  ⟷  add_label_dom (pfl, pf, y, y')"
\end{lstlisting}

\subsection{Congruence Closure Data Structure}
\label{subsection:datastructure}

For the congruence closure algorithm there are five important data structures, which are described in the following. More details on this topic can be found in \cite{Nieuwenhuis}.

\begin{itemize}
	\item \lstinline{cc_list}: the union-find list, corresponds to the \lstinline{uf_list}.

	\item \lstinline{use_list}: a two-dimensional list which contains for each representative $a$ a list of input equations $F(b_1, b_2) = b$ where the representative of $b_1$ or $b_2$ is $a$.

	\item \lstinline{lookup}: a lookup table indexed by pairs of representatives $b$ and $c$, which stores an input equation $F(a_1, a_2) = a$ such that $b$ is the representative of $a_1$ and $c$ is the representative of $a_2$, or \lstinline{None} if no such equation exists.

    \item \lstinline{pending}: equations of the type \lstinline{One} ($a = b$) or \lstinline{Two} ($F(a_1, a_2) = a$) ($F(b_1, b_2) = b$) where $a$ and $b$ need to be merged, and $a_1$ and $b_1$ are already in the same congruence class, as well as $a_2$ and $b_2$.

    \item \lstinline{proof_forest}: the proof forest as described in the previous subsection.

    \item \lstinline{pf_labels}: the labels of the proof forest as described in the previous subsection.

    \item \lstinline{input}: a set of the input equation, which will be useful for some proofs in the next sections.
\end{itemize}

In the following, we shall refer to \lstinline{cc_list} as $l$, the use list as $u$, the lookup table as $t$, the pending list as $pe$, the proof forest as $pf$, the labels list for the proof forest as $pfl$ and the input as $ip$, unless otherwise stated.

\subsection{Congruence Closure Algorithm}

With this data structure we can implement the \lstinline{merge} function as described in \cite{Nieuwenhuis}.
It takes as parameter the current congruence closure data structure and an equation which it adds to the data structure. It uses the \lstinline{propagate} function, which will be described later, which performs unions between the constant symbols in the pending list.

If the input equation is of the type $F(a_1, a_2) = a$, then there are two possibilities: if there is already an equation $F(b_1, b_2) = b$ in the lookup table at the index $(rep\_of(l,a_1), rep\_of(l,a_2))$, then we know that $a_1 = b_1$ and $a_2 = b_2$, and we add $F(b_1, b_2) = b$ and $F(a_1, a_2) = a$ to pending, so that the equivalence classes of $a$ and $b$ will be merged.
On the other hand, if the respective lookup entry is \lstinline{None}, then the equation is added to the lookup table, at the index $(rep\_of(l,a_1), rep\_of(l,a_2))$ so that the next time an equation with congruent parameters is input, they will be added together to pending.

For this case distinction there is a function \lstinline{lookup_Some}, which returns \lstinline{True} if there is an entry in lookup at the index $(rep\_of(l,a_1), rep\_of(l,a_2))$ and False otherwise, and a function \lstinline{update_lookup}, which adds the equation to lookup at the index $(rep\_of(l,a_1), rep\_of(l,a_2))$.

\begin{lstlisting}
fun merge :: "congruence_closure ⇒ equation ⇒ congruence_closure"
  where
"merge ⦇cc_list = l, use_list = u, lookup = t, pending = pe, proof_forest = pf, pf_labels = pfl, input = ip⦈
(a ≈ b) =
  propagate
    ⦇cc_list = l, use_list = u, lookup = t, pending = One (a ≈ b)#pe, proof_forest = pf, pf_labels = pfl, input = insert (a ≈ b) ip⦈"

| "merge ⦇cc_list = l, use_list = u, lookup = t, pending = pe, proof_forest = pf, pf_labels = pfl, input = ip⦈
(F a$_1$ a$_2$ ≈ a) =
(if (lookup_Some t l (F a$_1$ a$_2$ ≈ a))
  then propagate ⦇cc_list = l, use_list = u, lookup = t,
            pending = link_to_lookup t l (F a$_1$ a$_2$ ≈ a)#pe, proof_forest = pf, pf_labels = pfl, input = insert (F a$_1$ a$_2$ ≈ a) ip⦈
  else ⦇cc_list = l,
          use_list = (u[rep_of l a$_1$ := (F a$_1$ a$_2$ ≈ a)#(u ! rep_of l a$_1$)])[rep_of l a$_2$ := (F a$_1$ a$_2$ ≈ a)#(u ! rep_of l a$_2$)],
          lookup = update_lookup t l (F a$_1$ a$_2$ ≈ a),
          pending = pe, proof_forest = pf, pf_labels = pfl, input = insert (F a$_1$ a$_2$ ≈ a) ip⦈
)"
\end{lstlisting}

The main part of the algorithm is executed in \lstinline{propagate}, which recursively takes one item from pending and performs the union of the representative classes. As previously mentioned, the pending item could be either an equation of the type $a = b$, or two equations of the type $F(a_1, a_2) = a$ and $F(b_1, b_2) = b$, where $a_1$ and $a_2$ are already in the same representative class as $b_1$ and $b_2$ respectively. In both cases the representative classes of $a$ and $b$ need to be merged. The functions \lstinline{left} and \lstinline{right} simply retrieve $a$ and $b$ from either of the two types of pending equations. If $a$ and $b$ are already in the same representative class, nothing needs to be done, otherwise the union is performed. For more clarity, the union is defined separately as \lstinline{propagate_step}.

\begin{lstlisting}
function propagate :: "congruence_closure ⇒ congruence_closure"
  where
"propagate ⦇cc_list = l, use_list = u, lookup = t, pending = [], proof_forest = pf, pf_labels = pfl, input = ip⦈ =
⦇cc_list = l, use_list = u, lookup = t, pending = [], proof_forest = pf, pf_labels = pfl, input = ip⦈"
| "propagate
⦇cc_list = l, use_list = u, lookup = t, pending = (eq # pe), proof_forest = pf, pf_labels = pfl, input = ip⦈ =
(let a = left eq; b = right eq in
  (if rep_of l a = rep_of l b
    then propagate ⦇cc_list = l, use_list = u, lookup = t, pending = pe, proof_forest = pf, pf_labels = pfl, input = ip⦈
    else
      propagate (propagate_step l u t pe pf pfl ip a b eq)
))"
  by pat_completeness auto
\end{lstlisting}

The union consists of the previously discussed \lstinline{ufa_union}, \lstinline{add_edge} and \lstinline{add_label}, as well as a loop which moves all elements from the use list of $rep\_of(l,a)$ to either $rep\_of(l, b)$, or to pending. This is necessary, because the old representative of $a$ is not a representative any more, and its new representative is  $rep\_of(l, b)$.

\begin{lstlisting}
abbreviation propagate_step
  where
"propagate_step l u t pe pf pfl ip a b eq ≡
  propagate_loop (rep_of l b) (u ! rep_of l a)
    ⦇cc_list = ufa_union l a b,
    use_list = u[rep_of l a := []],
    lookup = t,
    pending = pe,
    proof_forest = add_edge pf a b,
    pf_labels = add_label pfl pf a eq,
    input = ip⦈"
\end{lstlisting}

The loop is defined as a recursive function, which considers each element of the use list of $rep\_of(l, a)$, and either adds it to the use list and the lookup table, or if there is already an entry in lookup, then that entry together with the current equation are added to pending.

\begin{lstlisting}
fun propagate_loop
  where
"propagate_loop rep_b (u1 # urest)
⦇cc_list = l, use_list = u, lookup = t, pending = pe, proof_forest = pf, pf_labels = pfl, input = ip⦈
=
  propagate_loop rep_b urest (
    if (lookup_Some t l u1)
    then
      ⦇cc_list = l, use_list = u, lookup = t,
            pending = link_to_lookup t l u1#pe,
            proof_forest = pf, pf_labels = pfl, input = ip⦈
    else
      ⦇cc_list = l,
            use_list = u[rep_b := u1 # (u ! rep_b)],
            lookup = update_lookup t l u1,
            pending = pe, proof_forest = pf, pf_labels = pfl, input = ip⦈
)"
| "propagate_loop _ [] cc = cc"
\end{lstlisting}

The function \lstinline{are_congruent} returns \lstinline{True} if an equation is in the congruence closure of all the input equations so far. It simply checks if the elements have the same representative or if they have the same representative as the corresponding entry in lookup.

\begin{lstlisting}
fun are_congruent :: "congruence_closure ⇒ equation ⇒ bool"
  where
"are_congruent ⦇cc_list = l, use_list = u, lookup = t, pending = pe, proof_forest = pf, pf_labels = pfl, input = ip⦈ (a ≈ b) =
    (rep_of l a = rep_of l b)"
| "are_congruent ⦇cc_list = l, use_list = u, lookup = t, pending = pe, proof_forest = pf, pf_labels = pfl, input = ip⦈ (F a$_1$ a$_2$ ≈ a) =
    (case lookup_entry t l a$_1$ a$_2$ of
      Some (F b$_1$ b$_2$ ≈ b) ⇒ (rep_of l a = rep_of l b)
    | None ⇒ False
)"
\end{lstlisting}

\section{Correctness Proof}

\subsection{Invariants}

At this point we can already prove some properties of the congruence closure data structure. Our approach this time is different than the one for the union-find algorithm. Instead of defining an induction rule like in the union-find section and then prove the properties through the induction rule, we define the properties as invariants and then prove that they remain invariant after applying merge. For each invariant, we need to follow the same steps. They are listed here in order to introduce the name for each step:

\begin{enumerate}
	\item Prove that the invariant holds for the initial empty congruence closure.

	\item Prove that if the invariant holds before the merge operation, it also holds after the merge operation. Below is a list of what needs to be proven:
    \begin{enumerate}
        \item The invariant holds after one step in the \lstinline{propagate_loop}. We shall refer to the two cases of the function as \lstinline{loop1} and \lstinline{loop2}.
    	\item The invariant holds after the entire \lstinline{propagate_loop}.
    	\item The invariant holds for the parameters of \lstinline{propagate_loop} in \lstinline{propagate_step}. We shall refer to the this case as \lstinline{mini_step}.
    	\item It holds after \lstinline{propagate_step}.
    	\item It holds after \lstinline{propagate}.
    	\item And finally, it holds after \lstinline{merge}.
    \end{enumerate}
\end{enumerate}

Let us now look at the concrete invariants. Each list in the data structure has an invariant which states that all the elements which are in the list are in bounds. The corresponding proofs are easy to prove if we assume that all the input equations contain only valid elements.

One of the invariants is the usual \lstinline{ufa_invar} that we know from the union-find algorithm. The \lstinline{ufa_invar} holds for the \lstinline{cc_list} and the \lstinline{proof_forest}. These two are only modified before entering the in the \lstinline{mini_step}, and we already proved previously that the \lstinline{ufa_invar} holds after \lstinline{ufa_union} (in Section \ref{section:uf-isabelle}) and after \lstinline{add_edge} (Subsection \ref{subsubsection:addedge}). Therefore it also holds after \lstinline{merge}.

We define a new invariant \lstinline{inv_same_rep_classes}, which states, as the name suggests, that the union-find forest and the proof forest represent the same equivalence classes:

\begin{lstlisting}
rep_of l i = rep_of l j ⟷ rep_of pf i = rep_of pf j
\end{lstlisting}

This invariant is important for the proofs that consider \lstinline{add_edge}. That is because we only showed that \lstinline{add_edge pf x y} terminates if \lstinline{rep_of pf x ≠ rep_of pf y}. \lstinline{propagate} only executes \lstinline{add_edge} when \lstinline{rep_of l x ≠ rep_of l y}. The invariant shows that these two statements are equivalent, therefore \lstinline{add_edge} always terminates when used inside of the algorithm.

In order to prove the invariant, given that the two lists $l$ and $pf$ are only modified during the \lstinline{mini_step}, it is sufficient to show that \lstinline{ufa_union l x y} and \lstinline{add_edge pf x y} have the same behaviour, which is what we showed in Subsection \ref{subsubsection:addedge}.

Furthermore, for each data structure there is an invariant which states the properties which were informally described in Subsection \ref{subsection:datastructure}. They are described in the following.

For the use list, the invariant \lstinline|use_list_invar| states that for each representative $a$, its use list only contains equations of the type $F(b_1, b_2) = b$, where $a$ is the representative of either $b_1$ or $b_2$.

\begin{proof}
For the correctness proof after the \lstinline{propagate_loop}, we need to add an additional assumption that the second parameter only contains equations of the type $F(a_1, a_2) = a$ and the representative of either $a_1$ or $a_2$ is $rep\_of(l, b)$ (where $b$ is the right side of the equation which is being propagated). This follows from the facts that the second parameter of \lstinline{propagate_loop} is $(u ! rep\_of(l, a))$, the invariant holds before the \lstinline{propagate_loop}, and the new representative of $a$ after the union is $b$.

With this assumption we can show that each time an equation gets added to the use list in the \lstinline{propagate_loop}, it is a valid equation.

In the proof after the \lstinline{merge} operation, the use list is only modified in the third case, and only equations of the form $F(a_1, a_2) = a$ are added to $rep\_of(l, a_1)$  and $rep\_of(l, a_2)$. Therefore all the necessary properties hold for these new equations.

For the remaining cases, use list is either unchanged, or something is removed from it, therefore the invariant trivially holds.
\end{proof}

The invariant \lstinline|lookup_invar| for lookup is similar, it states that each entry in the lookup table at index $(i, j)$, for representatives $i$ and $j$, is either \lstinline{None} or is an equation of the form $F(a_1, a_2) = a$ where the representative of $a_1$ is $i$ and the representative of $a_2$  is $j$.

\begin{proof}
Each time an equation is added to lookup, it has the desired form and it is added to the index $(rep\_of(l, a_1), rep\_of(l, a_2))$. This happens in the \lstinline{propagate_loop} and in \lstinline{merge}. In the \lstinline{propagate_loop}, the added equation derives from the use list, for which we proved with the previous invariant that its equations have the desired form. In merge, only the equations of the type $F(a_1, a_2) = a$ are added to lookup.
\end{proof}

For pending, the invariant \lstinline|pending_invar| states that the equations are either of the form \lstinline{One} ($a = b$) or \lstinline{Two} ($F(a_1, a_2) = a$) ($F(b_1, b_2) = b$) where $rep\_of(l, a_1) = rep\_of(l, b_1)$ and $rep\_of(l, a_2) = rep\_of(l, b_2)$:

\begin{proof}
We need to show that in the \lstinline{propagate_loop} the equation $u1$ we add to pending has a valid form. We know that $u1$ derives from the use list, therefore it is of the form $F(a_1, a_2) = a$. Then we link to it the lookup entry at the index $(rep\_of(l, a_1), rep\_of(l, a_2))$. From the lookup invariant we know that there is an entry of the form $F(b_1, b_2) = b$ at this index where $rep\_of(l, a_1) = rep\_of(l, b_1)$ and $rep\_of(l, a_2) = rep\_of(l, b_2)$. This shows that they are valid equations for pending.

The same holds for the equations added to pending in \lstinline{merge}.
\end{proof}

There is also an invariant which states that the \lstinline{cc_list}, the first dimension of the use list, both dimensions of lookup, the proof forest and the \lstinline{pf_labels} have the same length. This was trivial to prove, given that the algorithm never changes the length of the lists, and initially the lists have the same length.

The remaining invariants will be described later on, when they become relevant.

\subsection{Abstract Formalization of Congruence Closure}
\label{subsection:abstraction}

In order to prove the correctness of the algorithm, we define an abstraction of congruence closure. We cannot use any previously defined definitions, because the data structure that we use can only represent a subset of all possible equations, for example it cannot represent equations of the type $a = F(b, c)$ or $F (F (a, b), c) = d$. For this reason, we define an inductive set which represents the congruence closure of a set of equations and only uses our restricted definition of equation.

\begin{lstlisting}
inductive_set Congruence_Closure :: "equation set ⇒ equation set" for S
  where
    base: "eqt ∈ S ⟹ eqt ∈ Congruence_Closure S"
  | reflexive: "(a ≈ a) ∈ Congruence_Closure S"
  | symmetric: "(a ≈ b) ∈ Congruence_Closure S ⟹ (b ≈ a) ∈ Congruence_Closure S"
  | transitive1: "(a ≈ b) ∈ Congruence_Closure S ⟹ (b ≈ c) ∈ Congruence_Closure S
⟹ (a ≈ c) ∈ Congruence_Closure S"
  | transitive2: "(F a$_1$ a$_2$ ≈ b) ∈ Congruence_Closure S ⟹ (b ≈ c) ∈ Congruence_Closure S
⟹ (F a$_1$ a$_2$ ≈ c) ∈ Congruence_Closure S"
  | transitive3: "(F a$_1$ a$_2$ ≈ a) ∈ Congruence_Closure S
⟹ (a$_1$ ≈ b$_1$) ∈ Congruence_Closure S ⟹ (a$_2$ ≈ b$_2$) ∈ Congruence_Closure S
⟹ (F b$_1$ b$_2$ ≈ a) ∈ Congruence_Closure S"
  | monotonic: "(F a$_1$ a$_2$ ≈ a) ∈ Congruence_Closure S ⟹ (F a$_1$ a$_2$ ≈ b) ∈ Congruence_Closure S
⟹ (a ≈ b) ∈ Congruence_Closure S"
\end{lstlisting}

The following proof rule follows directly from the definition of congruence closure, and proved to be very useful for multiple proofs:

\begin{lstlisting}
lemma Congruence_Closure_eq[case_names left right]:
  assumes "⋀ a. a ∈ A ⟹ a ∈ Congruence_Closure B"
    "⋀ b. b ∈ B ⟹ b ∈ Congruence_Closure A"
  shows "Congruence_Closure A = Congruence_Closure B"
\end{lstlisting}

It is used to prove equality between congruence closures of $A$ and $B$. It states that it is sufficient to prove that all elements of set $A$ are in the congruence closure of $B$ and vice versa, instead of having to prove that all elements of the congruence closure of $A$ are in the congruence closure of $B$.

\subsection{Correctness}

To prove the correctness of the congruence closure implementation, we need to show that the invariants imply that \lstinline{are_congruent cc eq} returns \lstinline{True} if and only if the equation $eq$ lies in the congruence closure of the input equations.

\begin{lstlisting}
theorem are_congruenct_correct:
  assumes "cc_invar cc" "pending cc = []"
  shows "eq ∈ Congruence_Closure ((input cc)) ⟷ are_congruent cc eq"
\end{lstlisting}

The paper \cite{Nieuwenhuis} proves this by stating the folllowing invariant which holds throughout the algorithm,

\begin{lstlisting}
Congruence_Closure(representativeE ∪ pending) = Congruence_Closure (input)
\end{lstlisting}

where \lstinline{representativeE} can be seen as the set of equations derived from our union-find list and the equations in lookup. It is the union of the following two sets:

\begin{itemize}
    \item\lstinline{representative_set} is defined such that its congruence closure contains all the equations between two elements which have the same representative.
    \item\lstinline{lookup_entries_set} is the set of all the entries in lookup at indexes which are representatives.
\end{itemize}

\begin{lstlisting}
abbreviation representatives_set :: "nat list ⇒ equation set"
  where
    "representatives_set l ≡ {a ≈ rep_of l a |a. l ! a ≠ a}"
\end{lstlisting}

\begin{lstlisting}
abbreviation lookup_entries_set :: "congruence_closure ⇒ equation set"
  where
    "lookup_entries_set cc ≡ {F a' b' ≈ rep_of (cc_list cc) c | a' b' c c$_1$ c$_2$ .
                      cc_list cc ! a' = a' ∧ cc_list cc ! b' = b'
                      ∧ lookup cc ! a' ! b' = Some (F c$_1$ c$_2$ ≈ c)}"
\end{lstlisting}

\begin{lstlisting}
definition representativeE :: "congruence_closure ⇒ equation set"
  where
    "representativeE cc = representatives_set (cc_list cc) ∪ lookup_entries_set cc"
\end{lstlisting}

The formal definition of the aforementioned invariant is the following, where \lstinline{pending_set} converts the pending list to a set of equations of the type $a = b$:

\begin{lstlisting}
definition inv2 :: "congruence_closure ⇒ bool"
  where
    "inv2 cc ≡
Congruence_Closure (representativeE cc ∪ pending_set (pending cc)) = Congruence_Closure (input cc)"
\end{lstlisting}

The set of input equations is only modified by the \lstinline{merge} function, but remains constant throughout the \lstinline{propagate} function, therefore for the proof we just need to show that the congruence closure of the \lstinline{representativeE} set and pending remain unchanged after the \lstinline{propagate} function.

The main challenge is to prove that the invariant holds after the \lstinline{mini_step}. We will show that \lstinline{Congruence Closure (representativeE ∪ pending)} before the \lstinline{propagate_step} is equal to \lstinline{Congruence Closure (representativeE ∪ pending ∪ (u ! rep_of l a))} after the \lstinline{mini_step}.
Then we prove that the latter is equal to \lstinline{Congruence Closure (representativeE ∪ pending)}  after the \lstinline{propagate_loop}.
These two lemmas imply that the congruence closure of the \lstinline{representativeE} set and pending remain unchanged after the \lstinline{propagate} function.

We will first prove the second statement, given that it is much easer to show.

\begin{proof}
We need to show that \lstinline{Congruence Closure (representativeE ∪ pending ∪ (u ! rep_of l a))} before the \lstinline{propagate_loop} is equal to \lstinline{Congruence Closure (representativeE ∪ pending)}  after the \lstinline{propagate_loop}. In each step of the loop, one element from $(u ! rep\_of(l, a))$ is moved either to pending or to lookup. Therefore after the loop each element of $(u ! rep\_of(l, a))$ is either in pending or in representativeE. The elements of pending and representativeE are never removed from the set, therefore they are present also after the loop.
\end{proof}

The first lemma is more difficult to prove. The following is the statement of the lemma:

\begin{lstlisting}[label=lst:inv2_mini_step]
lemma inv2_mini_step:
  assumes  "a = left eq" "b = right eq"
  "cc_invar ⦇cc_list = l, use_list = u, lookup = t, pending = (eq # pe),
  proof_forest = pf, pf_labels = pfl, input = ip⦈"
     shows "Congruence_Closure
(representativeE
⦇cc_list = l, use_list = u, lookup = t, pending = (eq # pe),
proof_forest = pf, pf_labels = pfl, input = ip⦈
∪ pending_set (eq # pe))
=
Congruence_Closure (representativeE
⦇cc_list = ufa_union l a b,
    use_list = u[rep_of l a := []],
    lookup = t,
    pending = pe,
    proof_forest = add_edge pf a b,
    pf_labels = add_label pfl pf a eq,
    input = ip⦈
∪ pending_set pe
∪ set (u ! rep_of l a))"
\end{lstlisting}

\begin{proof}
There are two inclusions which need to be shown. We can use the rule \lstinline{Congruence_Closure_eq} from Subsection \ref{subsection:abstraction}, which means that it is sufficient to show that each equation in the set on the left-hand side is in the congruence closure of the right-hand side and vice versa.

"$\subseteq$" It needs to be shown that the equations of the \lstinline{representatives_set}, in lookup and in pending are in the Congruence Closure of the right-hand side.

Regarding the \lstinline{representatives_set}, all the elements which had the same representative before a union also have the same representative after a union.

For the pending set, we need to prove that the equation that is removed from pending is still in the congruence closure after the \lstinline{mini_step}. This holds, because the equation which is removed is $a = b$, and $a$ and $b$ are in the same equivalence class after the \lstinline{ufe_union}.

The problematic cases are the equations in lookup. Given that after the union $rep\_of(l, a)$ is not a root any more, the entries in lookup which have as first or second index $rep\_of(l, a)$ are not in the \lstinline{lookup_set} anymore after the union. The goal is to prove that these equations are exactly the equations which are present in $(u ! rep\_of(l, a))$, but until now it was only proven that the equations in the use list are valid, not that they are exhaustive. A new invariant \lstinline{use_list_inv2} is needed which states that all elements which are present in the lookup table at index $(i, j)$ are also in the corresponding use lists of $i$ and $j$. We will introduce this invariant later.

"$\supseteq$" We need to show that the equations of the \lstinline{representatives_set}, \lstinline{lookup_entries_set}, pending set and of $(u ! rep\_of(l, a))$ on the right-hand side are in the congruence closure of the left-hand side.

The \lstinline{representatives_set} contains equations of the type $c = rep\_of (ufa\_union(l, a, b), c)$ for each element $c$ which is not a root. If the representative after the union is the same before the union, the same equation is in the \lstinline{representatives_set} of the left-hand side. The only representative that is different than before the union is the representative of $a$, which has as new representative $rep\_of(l, b)$. The left-hand side contains the equations $b = rep\_of(l, b)$ and $a = b$ (which is in pending). By transitivity, the congruence closure also contains $a = rep\_of(l, b)$ and $rep\_of(l, b)$ is exactly the same as $rep\_of (ufa\_union(l, a, b), a)$.

Regarding lookup, all the elements which are roots after the union, are also roots before the union, therefore all elements in the \lstinline{lookup_entry_set} of the right-hand side are also in the left-hand side.

It is evident that the equations in pending on the right-hand side are also in pending in the left-hand side.

It is more difficult to show that the equations in $(u ! rep\_of(l, a))$ are also present in the lookup table of the left-hand side. As before, we need a new invariant \lstinline{lookup_invar2}, which states that all equations which are in the use list of a root $i$ are also present in the lookup table. Below is the description of the new invariants.
\end{proof}

We need two new invariants of this form:
\begin{itemize}
    \item \lstinline{lookup_invar2}: The elements in the lookup table are also present in the use list.
	\item \lstinline{use_list_invar2}: The elements in the use list are also present in the lookup table.
\end{itemize}

Unfortunately, these two invariants are not exactly true, because if there are two different equations where the elements have the same representatives, then they cannot both be present in the lookup table, because it only stores one equation for each pair of representatives. In fact, the set of equations in lookup and in the use list are not exactly the same, but for each equation in one of them, there is a ``similar'' equation in the other one.

The difficulty was to find a suitable definition of ``similar'' which is not too strong, otherwise it wouldn't be true, but also not too weak, otherwise it is not possible to prove the invariant \lstinline{inv2}.

The right definition of similar turned out to be the following:

\begin{definition}
Two equations $F(a_1, a_2) = a$ and $F(b_1, b_2) = b$ are \emph{similar}, if $rep\_of(l, a_1) = rep\_of(l, b_1)$, $rep\_of(l, a_2) = rep\_of(l, b_2)$ and $(a=b) \in Congruence\_Closure (representatives\_set \cup pending)$.
\end{definition}

Simply stating that $a$ and $b$ have the same representative would be too strong, because during the propagate function, they are added to pending in order to be merged later, and are not merged yet. If we use \lstinline{representativeE} instead of \lstinline{representatives_set}, the invariant is not strong enough in order to prove \lstinline{inv2_mini_step}.

The final invariants are the following:
\begin{itemize}
    \item \lstinline{lookup_invar2}: For each equation in lookup at the index $(i, j)$ (where $i$ and $j$ are representatives) there is a similar equation in use list $i$ and one in use list $j$.
	\item \lstinline{use_list_invar2}: For each equation $F(c_1, c_2) = c$ in use list at the index $i$ (where $i$ is a representative) there is a similar equation in lookup at the index $(rep\_of(l, c_1), rep\_of(l, c_2))$.
\end{itemize}

Here follows the proof for \lstinline{lookup_invar2}:

\begin{proof}
We need to show that if the invariant holds before \lstinline{merge}, then it also holds after the \lstinline{merge}.

The main aim is to show that it holds after \lstinline{propagate}. We assume that before \lstinline{propagate} the invariants \lstinline{lookup_invar2} and \lstinline{use_list_invar2} hold. In particular, we observe that for each equation $F(c_1, c_2) = c$ in $(u ! rep\_of(l, a))$ (which we shall refer to with $u_a$) there is a similar equation in lookup at the index $(rep\_of(l, c_1), rep\_of(l, c_2))$ and therefore there are similar equations in $(u ! rep\_of(l, c_1))$ and in $(u ! rep\_of(l, c_2))$.

In the \lstinline{propagate_loop} $u_a$ is emptied, while the other use lists are not modified, and $u_a$ is handed over as a parameter to the \lstinline{propagate_loop}.

From now on let $l$ be the \lstinline{cc_list} after the \lstinline{ufe_union}.

In \lstinline{loop1} the lookup table and the use lists are not modified, thus there is nothing to show.

In \lstinline{loop2} we take an equation $u1$ of the form $F(c_1, c_2) = c$ from $u_a$.

$u1$ is then added to lookup at the index $(rep\_of(l, c_1), rep\_of(l, c_2))$. We need to show that after this step, an equation similar to $u1$ is present both in the use list of $rep\_of(l, c_1)$ and the use list of $rep\_of(l, c_2)$.

This holds if none of those two use lists are $u_a$, because of the observation made earlier, and because no use list has not been modified apart from $u_a$.

If one (or both) of the use lists are $u_a$, then the representative of the corresponding element was $rep\_of(l,a)$ before the union, therefore after the union it is $rep\_of(l,b)$. Given that $u1$ is also added to $u ! rep\_of(l, b)$ by the function, we can conclude that there is a similar equation also in this use list.
\end{proof}

Here follows the proof for \lstinline{use_list_invar2}:

\begin{proof}
We need to show that if before the \lstinline{merge} the invariant holds, then it also holds after the \lstinline{merge}.

The difficulty in this proof was when after the \lstinline{mini_step}, because of the union, $rep\_of(l, a)$ is not a root any more, therefore if there are some equations $F(c_1, c_2) = c$ in the use list where the representative of $c_1$ or $c_2$ is $rep\_of(l,a)$, they have a new representative after the union, which means that the corresponding similar equations in the lookup table are not at the right index anymore.

If $F(c_1, c_2) = c$ was in $u ! rep\_of (l, a)$ (which we shall again refer to with $u_a$), then it is removed from the use list after the \lstinline|mini_step|, and the previous remark does not cause any problems. However, there could be equations in use list at an index which is not $rep\_of(l, a)$, where $c_1$ or $c_2$ has the same representative of $a$. These equations do not have a similar equation in lookup after the \lstinline{mini_step}, but we know from \lstinline{lookup_invar2} that they have a similar equation in $u_a$.
Therefore, in order show that \lstinline{use_list_invar2} holds after the \lstinline{propagate_loop}, it is sufficient to show that for each equation in $u_a$ a similar equation is added (or already present) in the lookup table after the \lstinline{propagate_loop}.

From now on let $l$ be the \lstinline{cc_list} after the \lstinline{ufe_union}.

The \lstinline{propagate_loop} removes an equation $F(c_1, c_2) = c$ from $u_a$, and it enters \lstinline{loop1} when lookup contains an equation $F(d_1, d_2) = d$ at the index $(rep\_of(l, c_1), rep\_of(l, c_2))$. Then the equation $c = d$ is added to pending. Note that $F(c_1, c_2) = c$ and $F(d_1, d_2) = d$ are similar at this point, because $rep\_of(l, c_1) = rep\_of(l, d_1)$ and $rep\_of(l, c_2) = rep\_of(l, d_2)$ follows from the \lstinline{lookup_invar}, and $c = d$ is added to pending.

This case is exactly the reason why we can only prove that there is a ``similar'' equation in lookup, and not exactly the same.

Regarding \lstinline{loop2}, it is entered when lookup contains \lstinline{None} at the index $(rep\_of(l, c_1), rep\_of(l, c_2))$. Then $F(c_1, c_2) = c$ is added to $(u ! rep\_of(l, b))$ and to lookup. Obviously, an equation is similar to itself, therfore after this lookup contains a similar equation to $F(c_1, c_2) = c$.

There is also a new element which is added to the use list, and it has a similar equation in lookup, which is $F(c_1, c_2) = c$ itself, therefore the invariant holds.
\end{proof}

With these two invariants the proof for \lstinline{inv2} is completed. Given that the pending list is always empty after the termination of propagate, the correctness proof \lstinline{are_congruent_correct}, which was defined at the beginning of this section, follows directly from this invariant.

All the invariants are put together in the invariant \lstinline{cc_invar}, and using all the previously described proofs about the invariant, we can prove that \lstinline{cc_invar} holds for the initial empty data structure, and that it holds after \lstinline{merge}.

\begin{lstlisting}
theorem cc_invar_initial_cc: "cc_invar (initial_cc n)"
\end{lstlisting}

\begin{proof}
All the above-mentioned invariants hold trivially for the initial case, given that all the data structures are empty or contain only \lstinline{None} in the beginning.
\end{proof}

\begin{lstlisting}
theorem cc_invar_merge:
  assumes "cc_invar cc" "valid_vars eq (nr_vars cc)"
  shows "cc_invar (merge cc eq)"
\end{lstlisting}

\begin{proof}
We already proved for each individual invariant, that they hold after propagate.
The proof for \lstinline|merge| uses the fact that \lstinline{propagate} terminates, which will be proven in the following section.
\end{proof}

\subsection{Termination}

We already proved that the functions \lstinline|add_edge| and \lstinline|add_label| terminate, the only missing proof is the termination of \lstinline|propagate|. All the remaining functions are simple enough for Isabelle to prove their termination automatically.

In order to prove the termination of \lstinline{propagate}, we show that the number of equivalence classes strictly decreases in each step of \lstinline{propagate}, therefore the function terminates at the latest when all the elements belong to the same equivalence class.

The number of equivalence classes is defined as the number of roots in the union-find forest. The function \lstinline|card| returns the cardinality of a set.

\begin{lstlisting}
abbreviation root_set
  where
    "root_set l ≡ {i | i. i < length l ∧ l ! i = i}"

definition nr_eq_classes :: "nat list ⇒ nat"
  where
    "nr_eq_classes l = card (root_set l)"
\end{lstlisting}

With this, we can show that after a union, $rep\_of (l, a)$ is not a root anymore, therefore there is one less root in the forest.

\begin{lstlisting}
lemma ufa_union_decreases_nr_eq_classes:
  assumes "ufa_invar l" "a < length l"
    "rep_of l a ≠ rep_of l b"
  shows "nr_eq_classes (ufa_union l a b) = nr_eq_classes l - 1"
\end{lstlisting}

The termination proof for \lstinline{propagate} follows from this lemma, with an additional assumption that there is at least one variable, so that there is at least one equivalence class.

\begin{lstlisting}
lemma propagate_domain:
  assumes "cc_invar cc" "nr_vars cc > 0"
  shows "propagate_dom cc"
\end{lstlisting}

\begin{proof}
We prove it by induction on the amount of equivalence classes in the union-find forest.

If the pending list is empty, the function terminates. Otherwise, the first element $eq$ is taken from the pending list. We define $a$ as \lstinline{left eq} and $b$ as \lstinline{right eq}, as in the function.

If $a$ and $b$ are already in the same equivalence class, then the union-find list is not modified, therefore we cannot use the induction hypothesis. Therefore we prove it by induction on the length of the pending list.

If $a$ and $b$ are not in the same equivalence class, they are merged by the \lstinline{propagate_step}, therefore the number of equivalence classes descreases by one according to the lemma \lstinline{ufa_union_decreases_nr_eq_classes} and we can prove the goal by using the induction hypothesis.
\end{proof}

\section{Implementation of the Explain Operation}

I implemented the explain operation for congruence closure, leaving the proof of termination and correctness open for future work. Nevertheless I would like to dedicate this section on the description of the implementation and a proposal of how the correctness could be proven.

TODO