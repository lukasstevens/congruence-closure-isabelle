% !TeX root = ../main.tex
% Add the above to each chapter to make compiling the PDF easier in some editors.

\chapter{Conclusion}\label{chapter:conclusion}

This thesis described the implementation of an \emph{explain} operation for the union-find algorithm implemented by Lammich \cite{unionfind-isabelle} in Isabelle/HOL. It can be used to generate certificates of equality for equations between constants. We formally proved that the algorithm terminates and returns a correct output. The implementation can be used independently of the congruence closure algorithm.

The paper \cite{Nieuwenhuis} shows the minimality of the set of equations produced by the \emph{explain} operation. For future work, this property could be formalized and verified in Isabelle.

The other main focus of this thesis was the implementation of the functions \lstinline|merge| and \lstinline|are_congruent|, as described in \cite{Nieuwenhuis}, which maintain the congruence closure of a set of equations. Invariants of the merge operation were identified and proven. The algorithm terminates and is sound and complete.

The \emph{explain} operation for congruence closure can produce certificates which validate the congruence of two terms. We proved the termination of the function and we discussed its correctness.

The formalization of the correctness proof for the \emph{explain} function is still open for future work.

The code of this thesis can be used in order to implement an automatic proof strategy for Isabelle/HOL. It can be refined by using the imperative code framework of Isabelle \cite{imperativehol} and the optimizations which were left out of this thesis can be included in the refinement. In order to use it as a proof strategy that works for arbitrary terms, it is also necessary to implement the initial transformations of the equations to the form used by this algorithm.

The applications which use the \emph{explain} operations for union-find and congruence closure usually need to reconstruct the proof for certain equations and they could do so more easily if the output of the \emph{explain} operations was not given in the form of a set of equations, but rather as a tree of equations. The refinement of the algorithm could modify the output format accordingly.
