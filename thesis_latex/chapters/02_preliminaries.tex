% !TeX root = ../main.tex
% Add the above to each chapter to make compiling the PDF easier in some editors.

\chapter{Preliminaries}\label{chapter:preliminaries}

\section{Isabelle/HOL}

Isabelle\cite{isabelle} is an automated theorem prover, which can be used among other things for the verification of algorithms. It provides different types of logic, the most used one being Higher-Order Logic (HOL), which is the one used in this thesis. For reasons of self-containedness of this thesis, this chapter introduces the notation used by Isabelle/HOL.

\subsubsection{Lists}

The syntax for the empty list in Isabelle is \lstinline|[]|, and longer lists are constructed with the infix \lstinline|#| operator, with which one element is appended to the front of the list. To concatenate two lists, the \lstinline|@| operator is used. \lstinline|set| is a function which converts a list to a set. Lists are indexed with the \lstinline|!| operator, and the syntax for updating a list $l$ at index $i$ is the following: \lstinline|l[i := new_value]|.

\subsubsection{Functions}

In Isabelle, the termination of functions must be proven. In the case of simple functions, Isabelle can prove it automatically. This is done for functions which are declared with the \lstinline{fun} keyword. For example the declaration \lstinline|fun f :: 'a ⇒ 'b| describes a function $f$ with a parameter of the type $'a$ and it returns a value of the type $'b$. Afterwards, the recursive equations of the function are defined.

Partial functions can also be defined in Isabelle, with the \lstinline{function} keyword.
The definition looks like this:

\begin{lstlisting}
function (domintros) g
  where "..."
  by pat_completeness auto
\end{lstlisting}

Isabelle automatically defines a predicate \lstinline|g_dom| where \lstinline{g_dom (a)} means that the function terminates with the parameter $a$.

The option \lstinline{domintros} provides inductive introduction rules for \lstinline|g_dom|, based on the defining equations of $g$.

After the function definition, it needs to be proven that the patterns used in the definition are complete and compatible. In our case, the method \lstinline|pat_completeness auto| always automatically proves this goal.

For a more detailed description of functions in Isabelle, see \cite{functions}.

\subsubsection{Records}

Records are similar to tuples, where each component has a name. For example, for the implementation of the union-find explain operations, we need three lists, which are grouped together in a record. The syntax is \lstinline{ufe = ⦇uf_list = l, unions = u, au = a⦈}, and to select for example the first component, we can write \lstinline{uf_list ufe}. The meaning of the three lists will be described in Section \ref{section:uf-data}.

For more information on records, see \cite{isabelle}, chapter 8.3.

\subsubsection{equivalence closure}

The theory ``Partial\_Equivalence\_Relation''\cite{Collections-AFP} defines the symmetric closure \lstinline{symcl} of a relation and the reflexiv-transitive closure is already part of the Isabelle/HOL distribution, and its syntax is \lstinline{R*}. The two definitions can be combined to have an abstraction of the equivalence closure. For example the equivalence closure of the relation $R$ is \lstinline{(symcl R)*}.

\subsubsection{Datatypes}
\label{subsubsection:datatypes}

New datatypes can be defined with the \lstinline|datatype| keyword. New datatypes consist of constructors and existing types. A concrete syntax for the new datatypes can be defined in brackets.

For example we define a new datatype for the two types of input equations used in the congruence closure algorithm. Equations of the type $a = b$ will be written as \lstinline{a ≈ b}, and equations of the type $F(a, b) = c$ are written as \lstinline{F a b ≈ c}.

\begin{lstlisting}
datatype equation = Constants nat nat ("_ ≈ _")
  | Function nat nat nat                 ("F _ _ ≈ _")
\end{lstlisting}

In this thesis, we will use the notation \lstinline{a ≈ b} and \lstinline{F a b ≈ c} in Isabelle listings, and $a = b$ and $F(a, b) = c$ outside of the listings.

\subsubsection{option}

The type \lstinline{option} models optional values. The value of a variable with type \lstinline{'a option} is either \lstinline|None| or \lstinline|Some a| where $a$ is a value with type $'a$.

\subsection{Related work}

Congruence closure exists in most automatic theorem provers, for example Coq has a \lstinline{congruence} tactic \cite{congruence-coq}, Lean also has a congruence-closure based decision procedure \cite{congruenceclosure-lean}
(cite: Z3, cvc4, coq congruence closure, lean itt)

