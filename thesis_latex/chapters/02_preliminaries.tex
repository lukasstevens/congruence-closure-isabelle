% !TeX root = ../main.tex
% Add the above to each chapter to make compiling the PDF easier in some editors.

\chapter{Preliminaries}\label{chapter:preliminaries}

\section{Isabelle/HOL}

 This chapter introduces the notation used by Isabelle/HOL.

\subsubsection{Lists}

The syntax for the empty list in Isabelle is \lstinline|[]|, and the infix operator \lstinline|#| is used to append one element to the front of the list.
The \lstinline|@| operator is used in order to concatenate two lists.
The function \lstinline|set| converts a list to a set.
Lists are indexed with the \lstinline|!| operator, and the syntax for updating a list $l$ at index $i$ is: \lstinline|l[i := new_value]|. The list \lstinline|[0..<n]| represents a list that contains all the numbers from $0$ to $n-1$.

\subsubsection{Functions}

In Isabelle, the termination of functions must be proven.
In the case of simple functions, Isabelle can prove it automatically.
This is done for functions which are declared with the \lstinline{fun} keyword.
For example, the declaration \lstinline|fun f :: 'a ⇒ 'b| describes a function $f$ with a parameter of the type $'a$ and it returns a value of the type $'b$.
Afterwards, the recursive equations of the function are defined.
Isabelle automatically defines induction rules for each function.

Partial functions can also be defined in Isabelle, with the \lstinline{function} keyword.
The definition looks like this:

\begin{lstlisting}
function (domintros) g
  where "..."
  by pat_completeness auto
\end{lstlisting}

Isabelle automatically defines a predicate \lstinline|g_dom| where \lstinline{g_dom(a)} means that the function \lstinline|g| terminates with the parameter $a$.
The option \lstinline{domintros} provides inductive introduction rules for \lstinline|g_dom|, based on the defining equations of $g$.

After the function definition, it needs to be proven that the patterns used in the definition are complete and compatible. In our case, the method \lstinline|pat_completeness auto| always automatically proves this goal.

Partial simplification rules and a partial induction rule are also automatically defined by Isabelle, but they can only be applied if we assume or prove that the function terminates with the given parameters.

For a more detailed description of functions in Isabelle, see \cite{functions}.

\subsubsection{Records}

Records are similar to tuples, where each component has a name. For example, for the implementation of the union-find \emph{explain} operation, we need three lists, called \lstinline|uf_list|, \lstinline|unions| and \lstinline|au|.
They are grouped together in the record \lstinline{ufe = ⦇uf_list = l, unions = u, au = a⦈}. In order to select, for example, the first component, we can write \lstinline{uf_list ufe}. The meaning of the three lists will be described in Section \ref{section:uf-data}.
For more information on records, see \cite{isabelle}, chapter 8.3.

\subsubsection{Equivalence Closure}

The theory ``Partial\_Equivalence\_Relation''\cite{Collections-AFP} defines the symmetric closure \lstinline{symcl} of a relation and the reflexiv-transitive closure is already part of the Isabelle/HOL distribution, and its syntax is \lstinline{R*}.
The two definitions can be combined to have an abstraction of the equivalence closure.
For example, the equivalence closure of the relation $R$ is \lstinline{(symcl R)*}.

\subsubsection{Datatypes}
\label{subsubsection:datatypes}

New datatypes can be defined with the \lstinline|datatype| keyword. New datatypes consist of constructors and existing types. A concrete syntax for the new datatypes can be defined in brackets.

For example, we define a new datatype for the two types of input equations used in the congruence closure algorithm. Equations of the type $a = b$ will be written as \lstinline{a ≈ b}, and equations of the type $F(a, b) = c$ are written as \lstinline{F a b ≈ c}.

\begin{lstlisting}
datatype equation = Constants nat nat ("_ ≈ _")
  | Function nat nat nat                 ("F _ _ ≈ _")
\end{lstlisting}

In this thesis, we will use the notation \lstinline{a ≈ b} and \lstinline{F a b ≈ c} in Isabelle listings, and $a = b$ and $F(a, b) = c$ outside of the listings.

\subsubsection{option}

The type \lstinline{option} models optional values. The value of a variable with type \lstinline{'a option} is either \lstinline|None| or \lstinline|Some x| where $x$ is a value with type $'a$. The function \lstinline|the| applied to \lstinline|Some x| returns \lstinline|x|, and it returns \lstinline|undefined| if the parameter is \lstinline|None|.

If \lstinline{'a} is an ordered type, the order is extended to \lstinline{'a option}, where \lstinline{None ≤ x} for all \lstinline|x|, and \lstinline{Some x ≤ Some y} iff \lstinline{x ≤ y}. This is defined in the Theory ``Option\_ord'' of the HOL library, which is included with the standard Isabelle/HOL distribution.

\subsection{Related work}

Efficient union-find algorithms have been known for a long time \cite{unionfind-og, Tarjan}. Given its importance as an algorithm, it was already formalized and verified in some of the most important theorem provers, such as Isabelle and Coq \cite{unionfind-persistent}. The code in this thesis uses the union-find formalization in Isabelle by Lammich, which was first published in a journal \cite{unionfind-isabelle} and later presented at a conference \cite{unionfind-isabelle-conference}. It includes the functions for \emph{union} and \emph{find}, as well as an invariant that characterizes the validity of the union-find data structure. It will be described in more detail in Section \ref{section:uf-isabelle}.

Based on the union-find implementation, efficient congruence closure algorithms have been developed by Shostak \cite{congruenceclosure-og2}, Nelson and Oppen \cite{congruenceclosure-og} and Downey et al. \cite{congruenceclosure-og3}.
Nieuwenhuis and Oliveras \cite{Nieuwenhuis} extended the algorithm by a \emph{cc\_explain} operation, which is necessary in the context of decision procedures, for example, for theorem provers.

Congruence closure is implemented in most automated theorem provers. Pierre Corbineau implemented a \lstinline|congruence| tactic for the theorem prover Coq \cite{congruenceclosure-coq, congruence-coq}, based on the algorithm of Downey et al. \cite{congruenceclosure-og3}, and with a \emph{cc\_explain} operation that is similar to the union-find \emph{explain} presented in this thesis and not as efficient as the \emph{cc\_explain} operation introduced by Nieuwenhuis et al. \cite{Nieuwenhuis}. He formally proved the correctness and termination of the congruence closure algorithm in the theorem prover Coq.

The theorem prover Lean also has a congruence closure-based decision procedure \cite{congruenceclosure-lean}, which is additionally able to handle dependent types. To my knowledge, it is not proven for correctness.

Isabelle/HOL includes a tool called \lstinline|sledgehammer| \cite{sledgehammer}, which uses external SMT solvers, e.g., Z3 \cite{z3} and CVC4 \cite{cvc4}, whose implementation is based on congruence closure.

However, there is to my knowledge no built-in congruence closure proof method for Isabelle yet.
The verified algorithms of this thesis can be used in the future in order to build such a proof method.
The thesis focuses on the formalization of the congruence closure algorithm in the functional programming language of Isabelle, leaving the implementation of the proof method open for future work.
For this, it will be necessary to implement an appropriate transformation of the input equations to the form used by our algorithm.
This transformation is described shortly in Section \ref{section:inputequations} and more in detail in \cite{Nieuwenhuis}, but the concrete implementation is not part of this thesis.
The optimizations that were omitted in this thesis should also be included in an implementation of the proof method.
