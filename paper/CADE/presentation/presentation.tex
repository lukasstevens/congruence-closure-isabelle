\documentclass[12pt, british,aspectratio=169]{beamer}
\usetheme[progressbar=foot]{metropolis}
\makeatletter
\setlength{\metropolis@progressonsectionpage@linewidth}{2.5pt}
\setlength{\metropolis@progressinheadfoot@linewidth}{2pt}
\makeatother

\usepackage{mathtools}
\usepackage{unicode-math}
\setmainfont{Fira Sans}
\setmathfont{STIX Two Math}

\usepackage[british]{babel}
\usepackage{csquotes}
\usepackage{appendixnumberbeamer}
\usepackage{booktabs}
\usepackage{bussproofs}
\usepackage{mathpartir}
\usepackage{forloop}
\usepackage{multirow}
\usepackage{graphicx}

%\usepackage[backend=bibtex, giveninits=true]{biblatex}
%\defbibenvironment{bibliography}
%  {\scriptsize\begin{itemize}}
%  {\end{itemize}}
%  {\item}%
%\addbibresource{../document/root.bib}

\usepackage{xpatch}
\makeatletter
\newcommand{\my@beamer@setsep}{%
\ifnum\@itemdepth=1\relax
     \setlength\itemsep{\my@beamer@itemsepi}% separation for first level
   \else
     \ifnum\@itemdepth=2\relax
       \setlength\itemsep{\my@beamer@itemsepii}% separation for second level
     \else
       \ifnum\@itemdepth=3\relax
         \setlength\itemsep{\my@beamer@itemsepiii}% separation for third level
   \fi\fi\fi}
\newlength{\my@beamer@itemsepi}\setlength{\my@beamer@itemsepi}{3ex}
\newlength{\my@beamer@itemsepii}\setlength{\my@beamer@itemsepii}{1.5ex}
\newlength{\my@beamer@itemsepiii}\setlength{\my@beamer@itemsepiii}{1.5ex}
\newcommand\setlistsep[3]{%
    \setlength{\my@beamer@itemsepi}{#1}%
    \setlength{\my@beamer@itemsepii}{#2}%
    \setlength{\my@beamer@itemsepiii}{#3}%
}
\xpatchcmd{\itemize}
  {\def\makelabel}
  {\my@beamer@setsep\def\makelabel}
 {}
 {}

\xpatchcmd{\beamer@enum@}
  {\def\makelabel}
  {\my@beamer@setsep\def\makelabel}
 {}
 {}
\makeatother


% Define TUM corporate design colors
% Taken from http://portal.mytum.de/corporatedesign/index_print/vorlagen/index_farben
\definecolor{TUMBlue}{HTML}{0065BD}
\definecolor{TUMSecondaryBlue}{HTML}{005293}
\definecolor{TUMSecondaryBlue2}{HTML}{003359}
\definecolor{TUMBlack}{HTML}{000000}
\definecolor{TUMWhite}{HTML}{FFFFFF}
\definecolor{TUMDarkGray}{HTML}{333333}
\definecolor{TUMGray}{HTML}{808080}
\definecolor{TUMLightGray}{HTML}{CCCCC6}
\definecolor{TUMAccentGray}{HTML}{DAD7CB}
\definecolor{TUMAccentOrange}{HTML}{E37222}
\definecolor{TUMAccentGreen}{HTML}{A2AD00}
\definecolor{TUMAccentLightBlue}{HTML}{98C6EA}
\definecolor{TUMAccentBlue}{HTML}{64A0C8}

\setbeamercolor{alerted text}{fg=TUMAccentOrange}
\setbeamercolor{title separator}{fg=TUMBlue}
\setbeamercolor{progress bar}{fg=TUMSecondaryBlue}
\setbeamercolor{frametitle}{bg=TUMBlue}

\usepackage{pgfplots,pgfplotstable}
\usepackage{tikz}
\usetikzlibrary{%
    shapes, shapes.symbols, shapes.geometric, arrows, arrows.meta, positioning,
    calc, shadows.blur, automata,
    decorations.pathreplacing, matrix,
    decorations.text, fit,
    patterns, patterns.meta,
    overlay-beamer-styles
}
\tikzset{
  >=stealth',
}
\tikzstyle{visalert on}=[visible on=<#1->, alt=<#1>{TUMAccentOrange}{}]
\tikzstyle{alert on}=[alt=<{#1}>{TUMAccentOrange}{}]
\tikzstyle{gray on}=[alt=<{#1}>{TUMLightGray}{}]
\usepackage{calc}
\usepackage{forest}


\usepackage[en-GB]{datetime2}

\usepackage{pgfpages}
\setbeameroption{%
  %show notes on second screen=right
}

\pgfdeclarelayer{background}
\pgfdeclarelayer{foreground}
\pgfsetlayers{background,main,foreground}

\usepackage[most]{tcolorbox} % Load tcolorbox with most features
\tcbset{
  rounded corners,
  colframe=TUMBlue,  % Border color
  boxrule=1pt,             % Thickness of the border
  arc=4mm,                 % Radius of the rounded corners
  center title,
}

% Defs
\directlua{
  vars = { "a", "b", "c", "d", "e", "f", "g" }
  eqs = { {1, 2}, {3, 4}, {4, 5}, {6, 2}, {7, 1}, {3, 2} }
}
\newcommand{\varN}[1]{\mathsf{\directlua{tex.sprint(vars[#1])}}}
\newcommand{\eqN}[1]{\directlua{%
    local eq = eqs[#1]
    tex.sprint("\\mathsf{" .. vars[eq[1]] .. "} \\thickapprox \\mathsf{" .. vars[eq[2]] .. "}")
}}

\newcommand{\explain}{\mathbfsf{explain}}
\newcommand{\explaine}{\mathbfsf{explain^\prime}}
\newcommand{\explaindom}{\mathbfsf{explain\_dom}}
\newcommand{\explainedom}{\mathbfsf{explain^\prime\_dom}}
\newcommand{\equivcl}{\mathbfsf{equivcl}}
\newcommand{\ufeunion}[1]{\mathbf{\cup}_{#1}}
\newcommand{\unions}{\mathbfsf{unions}}
\newcommand{\ufds}{\mathbfsf{union\_find}}
\newcommand{\auds}{\mathbfsf{assoc\_union}}

\title{Simplified and Verified: A Second Look at a Proof-Producing Union-Find Algorithm}
\subtitle{CADE 30}
\date{\DTMdisplaydate{2025}{7}{29}\\}
\author{\underline{Lukas Stevens}\inst{1} \and Rebecca Ghidini\inst{2,3}}
\institute{%
  \inst{1}Chair of LoVe, Technical University of Munich \\
  \inst{2}ENS, Département d’informatique, CNRS, PSL University \\
  \inst{3}Inria
}

%\makeatletter 
%\def\beamer@framenotesbegin{% at beginning of slide
%     \usebeamercolor[fg]{normal text}
%      \gdef\beamer@noteitems{}% 
%      \gdef\beamer@notes{}% 
%}
%\makeatother

\begin{document}

\maketitle
 
\section{Introduction}
% 
% \begin{frame}{Motivation}
%   \begin{columns}
%     \begin{column}{0.35\textwidth}
%       \begin{forest}
%         for tree={
%           math content,
%           grow'=north
%         }
%         [{a = b}, name=ab, visalert on={2}
%           [{a = d}, visalert on={4}, edge={visalert on={6}}
%             [{f(d) = a}, visible on=<1->, edge={visalert on={4}}]
%             [{f(d) = d}, visalert on={3}, edge={visalert on={4}}
%               [{f(b) = d}, visible on=<1->, edge={visalert on={3}}]
%               [{b = d}, name=bd, visible on=<1->, edge={visalert on={3}}]
%             ]
%           ]
%           [{}, no edge]
%         ]
%         \draw[visalert on={6}] (bd) to[out=-65, in=35] (ab);
%       \end{forest}
%     \end{column}
%     \hfill
%     \begin{column}{0.6\textwidth}
%       \begin{itemize}[<+(7)->]
%         \item Congruence closure ubiquitious in SMT
%         \item Equivalence closure is a special case 
%       \end{itemize}
%     \end{column}
%   \end{columns}
% \end{frame}
% 
% \begin{frame}{Goal}
%   \begin{columns}
%     \begin{column}{0.6\textwidth}
%   Verify a proof-producing union-find algorithm!\\[2em]
%   \visible<2->{But \textbf{why}?}
%     \end{column}
%     \hfill
%     \begin{column}{0.35\textwidth}
%       \includegraphics[width=\textwidth]{isabelle_ultras.png}
%     \end{column}
%   \end{columns}
% \end{frame}
% 
% \begin{frame}{Original paper}
%   \includegraphics[width=\textwidth]{original_paper.png}
% \end{frame}
% 
% \section{The algorithm(s)}
% 
% \begin{frame}{The union-find data structure}
% \textbf{Input:}
% \foreach \j [count=\i] in {2,...,7} {
%   \action<uncover@\j-|alert@\j>{$\eqN{\i}$}
% }
% \vspace{1em}
% \vfill
% \begin{columns}
% \begin{column}{0.2\textwidth}
% \begin{forest}
%   for tree={
%     math content,
%     draw,
%     circle,
%     minimum height=1.7em,
%     l=1.6cm,
%     s=1.4cm,
%     edge={<-}
%   }
%   [\varN{2}
%     [\varN{1}, edge={visalert on={2}}
%       [\varN{7}, edge={visalert on={6}}]]
%     [\varN{6}, edge={visalert on={5}}]
%     [\varN{5}, edge={visalert on={7}}
%       [\varN{4}, edge={visalert on={4}}
%         [\varN{3}, edge={visalert on={3}}]]]
%   ]
% \end{forest}
% \end{column}
% \hfill
% \begin{column}{0.6\textwidth}
%   \begin{itemize}[<+(7)->]
%     \item Forest of rooted trees
%     \item Tree $\hat{=}$ equivalence class
%     \item Root $\hat{=}$ representative
%   \end{itemize}
% \end{column}
% \end{columns}
% \end{frame}
% 
% 
% 
% \newcommand{\infline}[2]{([xshift=-0.2em,yshift=-0.25em]#1) -- ([xshift=0.2em,yshift=-0.25em]#2)}
% \begin{frame}{Proof format}
%   \begin{center}
%     \begin{tcolorbox}[width=0.75\textwidth, halign=center]
%     $\Gamma \vdash p : x \approx y$\pause{} $\quad\longleftarrow\:$ $p$ proves $x \approx y$ assuming $\Gamma$
%     \end{tcolorbox}
%     \vspace{1.5em}
%     \vfill
%     \begin{tikzpicture}[node distance=0cm, every node/.style={inner sep=0pt}]
%       \begin{scope}[visible on=<+(1)->]
%       \node (AssmC) {$\Gamma \vdash \mathsf{AssmP}~i : x \thickapprox y$};
%       \node[above=0.7em of AssmC.north, anchor=south] (AssmA) {$i < |\Gamma| \qquad \Gamma~!~i = x \thickapprox y$};
%       \draw[thick] \infline{AssmA.south west}{AssmA.south east};
%       \end{scope}
% 
%       \begin{scope}[visible on=<+(1)->]
%       \node[right=6cm of AssmC.south, anchor=south] (ReflC) {$\Gamma \vdash \mathsf{ReflP}~x : x \thickapprox x$};
%       \end{scope}
% 
%       \begin{scope}[visible on=<+(1)->]
%       \node[below left=2.5cm and 0cm of AssmC.south, anchor=south] (SymC) {$\Gamma \vdash \mathsf{SymP}~p : x \thickapprox y$};
%       \node[above=0.7em of SymC.north, anchor=south] (SymA) {$\Gamma \vdash p : y \thickapprox x$};
%       \path (SymC.west) |- coordinate (SymAL) (SymA.south west);
%       \path (SymC.east) |- coordinate (SymAR) (SymA.south east);
%       \draw[thick] \infline{SymAL}{SymAR};
%       \end{scope}
% 
%       \begin{scope}[visible on=<+(1)->]
%       \node[right=6cm of SymC.south, anchor=south] (TransC) {$\Gamma \vdash \mathsf{TransP}~p_1~p_2 : x \thickapprox z$};
%       \node[above=0.7em of TransC.north, anchor=south] (TransA) {$\Gamma \vdash p_1 : x \thickapprox y \qquad \Gamma \vdash p_2 : y \thickapprox z$};
%       \draw[thick] \infline{TransA.south west}{TransA.south east};
%       \end{scope}
%     \end{tikzpicture}
%   \end{center}
% \end{frame}
% 
% \begin{frame}{Soundness and Completeness of $\vdash$}
%   \textbf{Soundness + Completeness:}
%   \[(x, y) \in \equivcl~\Gamma \:\longleftrightarrow\: \exists p.\ \Gamma \vdash p : x \thickapprox y\]
% \end{frame}
% 
% \begin{frame}[standout]
%   \begin{center}
%   \LARGE How to combine union-find and proofs?
%   \end{center}
% \end{frame}

\begin{frame}{Associated unions}
\textbf{Input:}
\foreach \j [count=\i] in {2,...,7} {
  \action<uncover@\j-|alert@\j>{$\underbrace{\eqN{\i}}_\i$}
}
\vfill
\begin{columns}
\begin{column}{0.2\textwidth}
\begin{forest}
  for tree={
    math content,
    draw,
    circle,
    minimum height=1.7em,
    l=1.6cm,
    s=1.4cm,
    edge={<-}
  }
  [\varN{2}
    [\varN{1}, edge={visalert on={2}}, edge label={node[pos=0.6, above left] {$1$}}
      [\varN{7}, edge={visalert on={6}}, edge label={node[midway, left] {$5$}}]]
    [\varN{6}, edge={visalert on={5}}, edge label={node[pos=0.6, right] {$4$}}]
    [\varN{5}, edge={visalert on={7}}, edge label={node[pos=0.6, above right] {$6$}}
      [\varN{4}, edge={visalert on={4}}, edge label={node[midway, right] {$3$}}
        [\varN{3}, edge={visalert on={3}}, edge label={node[midway, right] {$2$}}]]]
  ]
\end{forest}
\end{column}
\begin{column}{0.8\textwidth}
\end{column}
\end{columns}
\end{frame}

% TODO: add paths to x and y?
% TODO: add animations?
% TODO: why most recent?
% TODO: highlight that this is the original algorithm, time travel one our algorithm
\begin{frame}{The efficient algorithm: idea}

  \begin{center}
  {%
  \Large How to explain that $x \thickapprox y$?
  }
  \vfill
  \begin{tikzpicture}[
    >=latex, node distance=0.5cm,
    aside/.append style={pattern=north east lines, pattern color=gray!55},
    bside/.append style={pattern={Dots[radius=0.65pt]}, pattern color=gray!55}
    ]
    \node[draw, circle, preaction={fill, white}, style=bside] (lca) {LCA};
    \begin{pgfonlayer}{background}
      \begin{scope}[shape=isosceles triangle, shape border rotate=90, minimum height=1.3cm, minimum width=2.2cm]
        \node[draw, anchor=north, below left=1.8cm and 3.5cm of lca.south west, aside] (l) {};
        \path (l.north) -- node[pos=0.54, draw, anchor=north, bside] (m) {} (lca.west);
        \node[draw, anchor=north, yshift=0.25cm, bside] (r) at (lca.south) {};
      \end{scope}
    \end{pgfonlayer}

    \node[draw, circle, above right=0.18cm and 0.05cm of l.south west] (x) {$x$};
    \node[draw, circle, left=0.1cm of l.east] (a) {$a$};
    \node[draw, circle, above left=of r.south east] (y) {$y$};
    \node[draw, circle, above right=of m.south west] (b) {$b$};
    
    \draw[->] (l.north) -- node[above, sloped] {$a \thickapprox b$} (m.north);
    \draw[->, dashed] (m.north) -- (lca.west);
    \draw[solid] (lca.north) -- ++(0,0.1);
    \draw[dashed] (lca.north) ++(0,0.1) -- ++(0,0.4);
  \end{tikzpicture}
  \end{center}
\end{frame}

% TODO: remove pedantic step 
% TODO: clarify transitivity
% TODO: clarify stripped down proofs
% TODO: Indicate where we are in the proof
% TODO: Highlight on which sides of the subgraphs the elements are (symmetry rule)
\begin{frame}{The efficient algorithm: example}
\textbf{Input:}
\foreach \al [count=\i] in {9,15,11,0,0,4} {
  \action<uncover@1-|alert@\al>{$\underbrace{\eqN{\i}}_\i$}
}
\vfill
\begin{columns}
\begin{column}{0.3\textwidth}
\begin{forest}
  for tree={
    math content,
    draw,
    circle,
    l=1.6cm,
    s=1.4cm,
    edge={<-}
  }
  [\varN{2}
    [\varN{1}, edge={alert on={3-9}}, edge label={node[pos=0.6, above left] {$1$}}
      [\varN{7}, edge={}, edge label={node[midway, left] {$5$}}]]
    [\varN{6}, edge={}, edge label={node[pos=0.6, right] {$4$}}]
    [\varN{5}, edge={alert on={3-7}}, edge label={node[pos=0.6, above right] {$6$}}
      [\varN{4}, edge={alert on={10-13}}, edge label={node[midway, right] {$3$}}
        [\varN{3}, edge={alert on={10-15}}, edge label={node[midway, right] {$2$}}]]]
  ]
\end{forest}
\end{column}
\hfill
\begin{column}{0.65\textwidth}
  \begin{forest}
    for tree={
      math content,
      grow'=north,
      edge path={},
      inner sep=0pt,
      s=0.2cm,
      s sep=0.6cm,
      l=0.5cm,
      l sep=0.3cm
    }
    [{\varN{1} \thickapprox \varN{5}}, visalert on={2}, name=15
    [{\varN{1} \thickapprox \varN{2}}, visalert on={6}, name=12] 
    [{\varN{2} \thickapprox \varN{3}}, visalert on={5}, name=23
    [{\varN{3} \thickapprox \varN{2}}, visalert on={4}, name=32]
      ]
      [{\varN{3} \thickapprox \varN{5}}, visalert on={7}, name=35
      [{\varN{3} \thickapprox \varN{4}}, visalert on={12}, name=34]
        [{\varN{4} \thickapprox \varN{5}}, visalert on={11}, name=45]
        [{\varN{5} \thickapprox \varN{5}}, visalert on={13}, name=55]
      ]
    ]
    \path let \p1 = (12.south west), \p2 = (15.north) in coordinate (l1l) at (\x1, {(\y1+\y2)/2});
    \path let \p1 = (35.south east), \p2 = (15.north) in coordinate (l1r) at (\x1, {(\y1+\y2)/2});
    \path let \p1 = (32.south west), \p2 = (23.north) in coordinate (l2l) at (\x1, {(\y1+\y2)/2});
    \path let \p1 = (32.south east), \p2 = (23.north) in coordinate (l2r) at (\x1, {(\y1+\y2)/2});
    \path let \p1 = (34.south west), \p2 = (35.north) in coordinate (l3l) at (\x1, {(\y1+\y2)/2});
    \path let \p1 = (55.south east), \p2 = (35.north) in coordinate (l3r) at (\x1, {(\y1+\y2)/2});
    \path let \p1 = (32.south), \p2 = (12.north west) in coordinate (l4l) at (\x2, {(\y1+\y2)/2});
    \path let \p1 = (32.south), \p2 = (12.north east) in coordinate (l4r) at (\x2, {(\y1+\y2)/2});
    \begin{scope}[thick]
      \draw[visible on=<3->, alert on={3-7}] (l1l) -- (l1r);
      \draw[visible on=<5->, alert on={5}] (l2l) -- (l2r);
      \draw[visible on=<10->, alert on={10-13}] (l3l) -- (l3r);
      \draw[visible on=<8->, alert on={8-9}] (l4l) -- coordinate (l4m) (l4r);
      \draw[visible on=<14->, alert on={14-15}] ([yshift=0.1cm]34.north west) -- ([yshift=0.1cm]34.north east);
      \draw[visible on=<17-18>, alert on={17}] ([yshift=0.1cm]55.north west) -- ([yshift=0.1cm]55.north east);
    \end{scope}
  \end{forest}
\end{column}
\end{columns}
\end{frame}

\begin{frame}{The efficient algorithm: recap}
\begin{columns}
  \begin{column}{0.3\textwidth}
    \scalebox{0.8}{%
  \begin{tikzpicture}[
    >=latex, node distance=0.5cm,
    aside/.append style={pattern=north east lines, pattern color=gray!55},
    bside/.append style={pattern={Dots[radius=0.65pt]}, pattern color=gray!55}
    ]
    \node[draw, circle, preaction={fill, white}, style=bside] (lca) {LCA};
    \begin{pgfonlayer}{background}
      \begin{scope}[shape=isosceles triangle, shape border rotate=90, minimum height=1.3cm, minimum width=2.2cm]
        \node[draw, anchor=north, below left=1.8cm and 3.5cm of lca.south west, aside] (l) {};
        \path (l.north) -- node[pos=0.54, draw, anchor=north, bside] (m) {} (lca.west);
        \node[draw, anchor=north, yshift=0.25cm, bside] (r) at (lca.south) {};
      \end{scope}
    \end{pgfonlayer}

    \node[draw, circle, above right=0.18cm and 0.05cm of l.south west] (x) {$x$};
    \node[draw, circle, left=0.1cm of l.east] (a) {$a$};
    \node[draw, circle, above left=of r.south east] (y) {$y$};
    \node[draw, circle, above right=of m.south west] (b) {$b$};
    
    \draw[->] (l.north) -- node[above, sloped] {$a \thickapprox b$} (m.north);
    \draw[->, dashed] (m.north) -- (lca.west);
    \draw[solid] (lca.north) -- ++(0,0.1);
    \draw[dashed] (lca.north) ++(0,0.1) -- ++(0,0.4);
  \end{tikzpicture}
    }
  \end{column}
  \hfill
  \begin{column}{0.65\textwidth}
    \begin{itemize}[<+(1)->]
      \item Exploits tree structure 
      \item Complex recursion 
    \end{itemize}
  \end{column}
\end{columns}
\end{frame}


%% TODO: remove pedantic step
% TODO: time travel joke? Use clock already?
\newcommand<>\gray[1]{%
  \textcolor#2{TUMLightGray}{#1}%
}
\begin{frame}{The simple algorithm: example}
\textbf{Input:}
\foreach \al/\g [count=\i] in {{18-19,34-35}/{36-},{16-17,28-33}/{18-20,34-},{14-15,25-27}/{16-20,28-},{12-13,23-24}/{14-20,25-},{10-11,21-22}/{12-20,23-},{1-9}/{10-}} {
  \action<uncover@1-|alert@\al|gray@\g>{$\underbrace{\eqN{\i}}_\i$}
}
\vfill
\begin{columns}
\begin{column}{0.3\textwidth}
\begin{forest}
  for tree={
    math content,
    draw,
    circle,
    l=1.6cm,
    s=1.4cm,
    edge={<-}
  }
  [\varN{2}
    [\varN{1}, edge={alert on={19-20,35}, gray on={36-}}, edge label={node[pos=0.6, above left] {$1$}}
      [\varN{7}, edge={alert on={11,22}, gray on={12-20,23-}}, edge label={node[midway, left] {$5$}}]]
    [\varN{6}, edge={alert on={13,24}, gray on={14-20,25-}}, edge label={node[pos=0.6, right] {$4$}}]
    [\varN{5}, edge={alert on={3-9}, gray on={10-}}, edge label={node[pos=0.6, above right] {$6$}}
      [\varN{4}, edge={alert on={15,26-27}, gray on={16-20,28-}}, edge label={node[midway, right] {$3$}}
        [\varN{3}, edge={alert on={17,29-33}, gray on={18-20,34-}}, edge label={node[midway, right] {$2$}}]]]
  ]
\end{forest}
\end{column}
\hfill
\begin{column}{0.65\textwidth}
  \begin{forest}
    for tree={
      math content,
      grow'=north,
      edge path={},
      inner sep=0pt,
      s=0.2cm,
      s sep=0.6cm,
      l=0.5cm,
      l sep=0.3cm
    }
    [{\varN{1} \thickapprox \varN{5}}, visalert on={2}, name=15
    [{\varN{1} \thickapprox \varN{2}}, visalert on={6}, name=12] 
      [{\varN{2} \thickapprox \varN{3}}, visalert on={5}, name=23
        [{\varN{3} \thickapprox \varN{2}}, visalert on={4}, name=32]
      ]
      [{\varN{3} \thickapprox \varN{5}}, visalert on={7}, name=35
        [{\varN{3} \thickapprox \varN{4}}, visalert on={28}, name=34]
        [{\varN{4} \thickapprox \varN{5}}, visalert on={27}, name=45]
        [{\varN{5} \thickapprox \varN{5}}, visalert on={29}, name=55]
      ]
    ]
    \path let \p1 = (12.south west), \p2 = (15.north) in coordinate (l1l) at (\x1, {(\y1+\y2)/2});
    \path let \p1 = (35.south east), \p2 = (15.north) in coordinate (l1r) at (\x1, {(\y1+\y2)/2});
    \path let \p1 = (32.south west), \p2 = (23.north) in coordinate (l2l) at (\x1, {(\y1+\y2)/2});
    \path let \p1 = (32.south east), \p2 = (23.north) in coordinate (l2r) at (\x1, {(\y1+\y2)/2});
    \path let \p1 = (34.south west), \p2 = (35.north) in coordinate (l3l) at (\x1, {(\y1+\y2)/2});
    \path let \p1 = (55.south east), \p2 = (35.north) in coordinate (l3r) at (\x1, {(\y1+\y2)/2});
    \path let \p1 = (32.south), \p2 = (12.north west) in coordinate (l4l) at (\x2, {(\y1+\y2)/2});
    \path let \p1 = (32.south), \p2 = (12.north east) in coordinate (l4r) at (\x2, {(\y1+\y2)/2});
    \begin{scope}[thick]
      \draw[visible on=<3->, alert on={3-8}] (l1l) -- (l1r);
      \draw[visible on=<5->] (l2l) -- (l2r);
      \draw[visible on=<21->, alert on={21-29}] (l3l) -- (l3r);
      \draw[visible on=<9->, alert on={9-20}] (l4l) -- coordinate (l4m) (l4r);
      \draw[visible on=<30->, alert on={30-31}] ([yshift=0.1cm]34.north west) -- ([yshift=0.1cm]34.north east);
      \draw[visible on=<32->, alert on={32-35}] ([yshift=0.1cm]55.north west) -- ([yshift=0.1cm]55.north east);
    \end{scope}
  \end{forest}
\end{column}
\end{columns}
\end{frame}

\begin{frame}{The simple algorithm: recap}
\begin{columns}
  \begin{column}{0.3\textwidth}
    \begin{tikzpicture}[scale=1.5, >=Stealth]

  % Clock circle
  \draw[thick] (0,0) circle (1cm);

  % Hour markers
  \foreach \angle in {0,30,...,330} {
    \draw[thick] ({cos(\angle)}, {sin(\angle)}) -- ({0.9*cos(\angle)}, {0.9*sin(\angle)});
  }

  % Clock hands (e.g., 10:10)
  \draw[very thick] (0,0) -- ({0.7*cos(60)}, {0.7*sin(60)}); % Minute hand
  \draw[very thick] (0,0) -- ({0.45*cos(120)}, {0.45*sin(120)}); % Hour hand

  % Counter-clockwise arrows around the clock
  \foreach \angle in {60} {
    \draw[->, thick, TUMAccentOrange, decorate, decoration={snake, amplitude=0.5mm, segment length=2mm, post length=7pt}]
      ({1.2*cos(\angle)}, {1.2*sin(\angle)}) 
      arc[start angle=\angle, end angle=\angle+60, radius=1.2cm];
  }

\end{tikzpicture}
  \end{column}
  \hfill
  \begin{column}{0.6\textwidth}
    \begin{itemize}[<+(1)->]
      \item Recursion on input list 
      \item Independent of tree structure
    \end{itemize}
  \end{column}
\end{columns}
\end{frame}



%\section{Correctness of the algorithm(s)}
%
%% TODO: Dr. Who?
%\begin{frame}{Who is who of the algorithms}
%  \begin{itemize}[<+(1)->]
%    \item $\explain$: the simple one
%    \item $\explaine$: the efficient one
%    \item $\explaindom$ and $\explainedom$: characterise termination
%  \end{itemize}
%\end{frame}
%
%\begin{frame}{Operations on union-find-explain}
%\textbf{Input:}
%\foreach \al [count=\i] in {{2,3},{2,3},{2,3},{2,3},{2,3},{2,3,6}} {
%  \action<alert@\al>{$\underbrace{\eqN{\i}}_\i$}
%}
%\vspace{1em}
%\vfill
%\begin{columns}
%\begin{column}{0.2\textwidth}
%\begin{forest}
%  for tree={
%    math content,
%    draw,
%    circle,
%    {alert on={{2,4}}},
%    l=1.6cm,
%    s=1.4cm,
%    edge={<-, alert on={{2,4}}},
%  }
%  [\varN{2}
%    [\varN{1}, edge label={node[text=black, alert on={{2,5}}, pos=0.6, above left] {$1$}}
%      [\varN{7}, edge label={node[text=black, alert on={{2,5}}, midway, left] {$5$}}]]
%    [\varN{6}, edge label={node[text=black, alert on={{2,5}}, pos=0.6, right] {$4$}}]
%    [\varN{5}, edge={alert on={{2,6}}}, edge label={node[text=black, alert on={{2,5,6}}, pos=0.6, above right] {$6$}}
%      [\varN{4}, edge label={node[text=black, alert on={{2,5}}, midway, right] {$3$}}
%        [\varN{3}, edge label={node[text=black, alert on={{2,5}}, midway, right] {$2$}}]]]
%  ]
%\end{forest}
%\end{column}
%\hfill
%\begin{column}{0.65\textwidth}
%  \visible<2->{Let \alert<2>{$ufe$} be the whole data structure}\\[0.7em]
%  \begin{itemize}[<+(2)-|alert@+(2)>]
%    \setlength{\itemsep}{1em}
%    \item $\unions~ufe$
%    \item $\ufds~ufe$
%    \item $\auds~ufe~x$
%    \item $\ufeunion{ufe}~x~y$
%  \end{itemize}
%\end{column}
%\end{columns}
%\end{frame}
%
%\begin{frame}{Correctness of $\explain$}
%  \begin{itemize}[<+(1)->]
%    \item \textbf{Soundness:} $\vdash$ is sound
%    \item \textbf{Completeness:} 
%      $(x, y) \in \equivcl~(\unions~ufe) \longrightarrow \unions~ufe \vdash \explain~ufe~x~y : x \thickapprox y$
%    \item \textbf{Termination:} $\explaindom~ufe~(x, y)$ 
%  \end{itemize}
%\end{frame}
%
%\begin{frame}{Correctness of $\explaine$}
%  \begin{itemize}
%    \item<1-> \textbf{Termination:} $\explainedom~ufe~(x, y)$
%      \begin{itemize}
%        \item<3-> $\explainedom~ufe~(x, y) \longrightarrow \explainedom~(\ufeunion{ufe}~a~b)~(x, y)$
%      \end{itemize}
%    \item<2-> \textbf{Soundness + Completeness:} $\explaine~ufe~x~y = \explain~ufe~x~y$
%      \begin{itemize}
%        \item<4-> Invariance of helpers under $\ufeunion{ufe}$
%        \item<5-> $\explaine~(\ufeunion{ufe}~a~b)~x~y = \explaine~ufe~x~y$
%      \end{itemize}
%  \end{itemize}
%\end{frame}
%
%\section{Refinement to efficient code}
%
%% TODO: more explanation what we are doing here. Why can't we execute the code as is? What is refinement?
%\begin{frame}{The refinement process}
%  \begin{enumerate}[<+(1)->]
%    \item Functional refinements
%    \item Refinement to Imperative HOL (using Lammich's Separation Logic Framework)
%    \item Export imperative code to e.g.\ Standard ML
%  \end{enumerate}
%\end{frame}
%
%% TODO: why a list of natural numbers?
%\begin{frame}{The data structure, abstractly}
%  \begin{tikzpicture}[every node/.style={inner sep=2pt}]
%    \node (unions) {$\unions~ufe$};
%    \node[base right=0.4cm of unions, anchor=base west] (unionsty) {$(\mathbb{N} \times \mathbb{N})~\mathsf{list}$};
%    \path (unions.east) -- node[midway] {$:$} (unionsty.west);
%
%    \node[below=4em of unions.west, anchor=west] (ufds) {$\ufds~ufe$};
%    \node[base right=0.4cm of ufds, anchor=base west] (ufdsty) {$\mathbb{N}~\mathsf{list}$};
%    \path (ufds.east) -- node[midway] {$:$} (ufdsty.west);
%
%    \node[below=4em of ufds.west, anchor=west] (auds) {$\auds~ufe$};
%    \node[base right=0.4cm of auds, anchor=base west] (audsty) {$\mathbb{N} \rightarrow \mathbb{N}~\mathsf{option}$};
%    \path (auds.east) -- node[midway] {$:$} (audsty.west);
%  \end{tikzpicture}
%  \vfill
%\end{frame}
%
%% TODO: ajdust arrows?
%% TODO: note capacity, note dynamic growth
%\begin{frame}{Refining $\unions$}
%  \begin{tikzpicture}[every node/.style={inner sep=2pt}]
%    \node (unions) {$\unions~ufe$};
%    \node[base right=0.4cm of unions, anchor=base west] (unionsty) {$(\mathbb{N} \times \mathbb{N})~\mathsf{list}$};
%    \path (unions.east) -- node[midway] {$:$} (unionsty.west);
%
%    \begin{scope}[visible on=<2->]
%    \node[below=1cm of unionsty.west, anchor=west] (unionsty1) {$(\mathbb{N} \times \mathbb{N})~\mathsf{array\_list}$};
%    \draw[->] ([xshift=0.7cm]unionsty.south west) -- ([xshift=0.7cm]unionsty1.north west);
%    \end{scope}
%
%    \begin{scope}[visible on=<3->]
%    \node[base right=0.7cm of unionsty1, anchor=base west] (unionsty2) {$(\mathbb{N} \times \mathbb{N})~\mathsf{array} \times \mathbb{N}$};
%    \path (unionsty1) -- node {$=$} (unionsty2);
%    \end{scope}
%  \end{tikzpicture}
%  \vfill
%\end{frame}
%
%% TODO: note rank, path compression
%\begin{frame}{Refining $\ufds$}
%  \begin{tikzpicture}[every node/.style={inner sep=2pt}]
%    \node[below=4em of unions.west, anchor=west] (ufds) {$\ufds~ufe$};
%    \node[base right=0.4cm of ufds, anchor=base west] (ufdsty) {$\mathbb{N}~\mathsf{list}$};
%    \path (ufds.east) -- node[midway] {$:$} (ufdsty.west);
%
%    \begin{scope}[visible on=<2->]
%    \node[below=1cm of ufdsty.west, anchor=west] (ufdsty1) {$\mathbb{N}~\mathsf{list} \times (\mathbb{N} \rightarrow \mathbb{N})$};
%    \draw[->] ([xshift=0.6cm]ufdsty.south west) -- ([xshift=0.6cm]ufdsty1.north west);
%    \end{scope}
%
%    \begin{scope}[visible on=<3->]
%    \node[below=1cm of ufdsty1.west, anchor=west] (ufdsty2) {$\mathbb{Z}~\mathsf{list}$};
%    \draw[->] ([xshift=0.6cm]ufdsty1.south west) -- ([xshift=0.6cm]ufdsty2.north west);
%    \end{scope}
%
%    \begin{scope}[visible on=<4->]
%    \node[below=1cm of ufdsty2.west, anchor=west] (ufdsty3) {$\mathbb{Z}~\mathsf{array}$};
%    \draw[->] ([xshift=0.6cm]ufdsty2.south west) -- ([xshift=0.6cm]ufdsty3.north west);
%    \end{scope}
%
%    \begin{scope}[visible on=<5->]
%    \node[base right=0.5cm of ufdsty3, anchor=base west] (ufdsty4) {$\mathbb{Z}~\mathsf{array}$};
%    \draw[->] (ufdsty2.east) -| (ufdsty4);
%    \end{scope}
%  \end{tikzpicture}
%  \vfill
%\end{frame}
%
%\begin{frame}{Refining $\auds$}
%  \begin{tikzpicture}[every node/.style={inner sep=2pt}]
%    \node[below=4em of ufds.west, anchor=west] (auds) {$\auds~ufe$};
%    \node[base right=0.4cm of auds, anchor=base west] (audsty) {$\mathbb{N} \rightarrow \mathbb{N}~\mathsf{option}$};
%    \path (auds.east) -- node[midway] {$:$} (audsty.west);
%
%    \begin{scope}[visible on=<2->]
%    \node[below=1cm of audsty.west, anchor=west] (audsty1) {$(\mathbb{N}~\mathsf{option})~\mathsf{array}$};
%    \draw[->] ([xshift=0.7cm]audsty.south west) -- ([xshift=0.7cm]audsty1.north west);
%    \end{scope}
%  \end{tikzpicture}
%  \vfill
%\end{frame}
%
%\section{Conclusion}
%
%%\begin{frame}{Benchmarks}
%%  Show benchmarks, explain outliers
%%\end{frame}
%
%% TODO: Note benchmarks
%
%\begin{frame}{Future work}
%  \begin{itemize}[<+(1)->]
%    \item Second variant of $\mathbfsf{union}$/$\explain$
%    \item Congruence closure
%    \item Nelson-Oppen theory combination
%  \end{itemize}
%\end{frame}
%
%\begin{frame}[standout]
%  \LARGE Thank you for listening!
%  \pause\vfill
%  Questions?
%\end{frame}

\end{document}

