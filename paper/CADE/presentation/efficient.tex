% TODO: add paths to x and y?
% TODO: add animations?
% TODO: why most recent?
% TODO: highlight that this is the original algorithm, time travel one our algorithm
\begin{frame}{The efficient algorithm: idea}

  \begin{center}
  {%
  \Large How to explain that $x \thickapprox y$?
  }
  \vfill
  \begin{tikzpicture}[
    >=latex, node distance=0.5cm,
    aside/.append style={pattern=north east lines, pattern color=gray!55},
    bside/.append style={pattern={Dots[radius=0.65pt]}, pattern color=gray!55}
    ]
    \node[draw, circle, preaction={fill, white}, style=bside] (lca) {LCA};
    \begin{pgfonlayer}{background}
      \begin{scope}[shape=isosceles triangle, shape border rotate=90, minimum height=1.3cm, minimum width=2.2cm]
        \node[draw, anchor=north, below left=1.8cm and 3.5cm of lca.south west, aside] (l) {};
        \path (l.north) -- node[pos=0.54, draw, anchor=north, bside] (m) {} (lca.west);
        \node[draw, anchor=north, yshift=0.25cm, bside] (r) at (lca.south) {};
      \end{scope}
    \end{pgfonlayer}

    \node[draw, circle, above right=0.18cm and 0.05cm of l.south west] (x) {$x$};
    \node[draw, circle, left=0.1cm of l.east] (a) {$a$};
    \node[draw, circle, above left=of r.south east] (y) {$y$};
    \node[draw, circle, above right=of m.south west] (b) {$b$};
    
    \draw[->] (l.north) -- node[above, sloped] {$a \thickapprox b$} (m.north);
    \draw[->, dashed] (m.north) -- (lca.west);
    \draw[solid] (lca.north) -- ++(0,0.1);
    \draw[dashed] (lca.north) ++(0,0.1) -- ++(0,0.4);
  \end{tikzpicture}
  \end{center}
\end{frame}

% TODO: remove pedantic step 
% TODO: clarify transitivity
% TODO: clarify stripped down proofs
% TODO: Indicate where we are in the proof
% TODO: Highlight on which sides of the subgraphs the elements are (symmetry rule)
\begin{frame}{The efficient algorithm: example}
\textbf{Input:}
\foreach \al [count=\i] in {9,15,11,0,0,4} {
  \action<uncover@1-|alert@\al>{$\underbrace{\eqN{\i}}_\i$}
}
\vfill
\begin{columns}
\begin{column}{0.3\textwidth}
\begin{forest}
  for tree={
    math content,
    draw,
    circle,
    l=1.6cm,
    s=1.4cm,
    edge={<-}
  }
  [\varN{2}
    [\varN{1}, edge={alert on={3-9}}, edge label={node[pos=0.6, above left] {$1$}}
      [\varN{7}, edge={}, edge label={node[midway, left] {$5$}}]]
    [\varN{6}, edge={}, edge label={node[pos=0.6, right] {$4$}}]
    [\varN{5}, edge={alert on={3-7}}, edge label={node[pos=0.6, above right] {$6$}}
      [\varN{4}, edge={alert on={10-13}}, edge label={node[midway, right] {$3$}}
        [\varN{3}, edge={alert on={10-15}}, edge label={node[midway, right] {$2$}}]]]
  ]
\end{forest}
\end{column}
\hfill
\begin{column}{0.65\textwidth}
  \begin{forest}
    for tree={
      math content,
      grow'=north,
      edge path={},
      inner sep=0pt,
      s=0.2cm,
      s sep=0.6cm,
      l=0.5cm,
      l sep=0.3cm
    }
    [{\varN{1} \thickapprox \varN{5}}, visalert on={2}, name=15
    [{\varN{1} \thickapprox \varN{2}}, visalert on={6}, name=12] 
    [{\varN{2} \thickapprox \varN{3}}, visalert on={5}, name=23
    [{\varN{3} \thickapprox \varN{2}}, visalert on={4}, name=32]
      ]
      [{\varN{3} \thickapprox \varN{5}}, visalert on={7}, name=35
      [{\varN{3} \thickapprox \varN{4}}, visalert on={12}, name=34]
        [{\varN{4} \thickapprox \varN{5}}, visalert on={11}, name=45]
        [{\varN{5} \thickapprox \varN{5}}, visalert on={13}, name=55]
      ]
    ]
    \path let \p1 = (12.south west), \p2 = (15.north) in coordinate (l1l) at (\x1, {(\y1+\y2)/2});
    \path let \p1 = (35.south east), \p2 = (15.north) in coordinate (l1r) at (\x1, {(\y1+\y2)/2});
    \path let \p1 = (32.south west), \p2 = (23.north) in coordinate (l2l) at (\x1, {(\y1+\y2)/2});
    \path let \p1 = (32.south east), \p2 = (23.north) in coordinate (l2r) at (\x1, {(\y1+\y2)/2});
    \path let \p1 = (34.south west), \p2 = (35.north) in coordinate (l3l) at (\x1, {(\y1+\y2)/2});
    \path let \p1 = (55.south east), \p2 = (35.north) in coordinate (l3r) at (\x1, {(\y1+\y2)/2});
    \path let \p1 = (32.south), \p2 = (12.north west) in coordinate (l4l) at (\x2, {(\y1+\y2)/2});
    \path let \p1 = (32.south), \p2 = (12.north east) in coordinate (l4r) at (\x2, {(\y1+\y2)/2});
    \begin{scope}[thick]
      \draw[visible on=<3->, alert on={3-7}] (l1l) -- (l1r);
      \draw[visible on=<5->, alert on={5}] (l2l) -- (l2r);
      \draw[visible on=<10->, alert on={10-13}] (l3l) -- (l3r);
      \draw[visible on=<8->, alert on={8-9}] (l4l) -- coordinate (l4m) (l4r);
      \draw[visible on=<14->, alert on={14-15}] ([yshift=0.1cm]34.north west) -- ([yshift=0.1cm]34.north east);
      \draw[visible on=<17-18>, alert on={17}] ([yshift=0.1cm]55.north west) -- ([yshift=0.1cm]55.north east);
    \end{scope}
  \end{forest}
\end{column}
\end{columns}
\end{frame}

\begin{frame}{The efficient algorithm: recap}
\begin{columns}
  \begin{column}{0.3\textwidth}
    \scalebox{0.8}{%
  \begin{tikzpicture}[
    >=latex, node distance=0.5cm,
    aside/.append style={pattern=north east lines, pattern color=gray!55},
    bside/.append style={pattern={Dots[radius=0.65pt]}, pattern color=gray!55}
    ]
    \node[draw, circle, preaction={fill, white}, style=bside] (lca) {LCA};
    \begin{pgfonlayer}{background}
      \begin{scope}[shape=isosceles triangle, shape border rotate=90, minimum height=1.3cm, minimum width=2.2cm]
        \node[draw, anchor=north, below left=1.8cm and 3.5cm of lca.south west, aside] (l) {};
        \path (l.north) -- node[pos=0.54, draw, anchor=north, bside] (m) {} (lca.west);
        \node[draw, anchor=north, yshift=0.25cm, bside] (r) at (lca.south) {};
      \end{scope}
    \end{pgfonlayer}

    \node[draw, circle, above right=0.18cm and 0.05cm of l.south west] (x) {$x$};
    \node[draw, circle, left=0.1cm of l.east] (a) {$a$};
    \node[draw, circle, above left=of r.south east] (y) {$y$};
    \node[draw, circle, above right=of m.south west] (b) {$b$};
    
    \draw[->] (l.north) -- node[above, sloped] {$a \thickapprox b$} (m.north);
    \draw[->, dashed] (m.north) -- (lca.west);
    \draw[solid] (lca.north) -- ++(0,0.1);
    \draw[dashed] (lca.north) ++(0,0.1) -- ++(0,0.4);
  \end{tikzpicture}
    }
  \end{column}
  \hfill
  \begin{column}{0.65\textwidth}
    \begin{itemize}[<+(1)->]
      \item Exploits tree structure 
      \item Complex recursion 
    \end{itemize}
  \end{column}
\end{columns}
\end{frame}
